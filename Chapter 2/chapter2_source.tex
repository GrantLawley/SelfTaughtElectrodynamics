\documentclass{article}
\usepackage[utf8]{inputenc}
\usepackage{graphicx} % Required for inserting images
\usepackage{tikz}
\usetikzlibrary{angles, quotes}
\usepackage{enumitem}
\usepackage{tikz-3dplot}
\usepackage{parskip}
\usepackage{amsmath, amssymb, amsfonts}
\usetikzlibrary{arrows.meta}
\newcommand{\bvec}[1]{\mathbf{\hat{#1}}}
\newcommand{\delx}{\frac{\partial}{\partial x}}
\newcommand{\dely}{\frac{\partial}{\partial y}}
\newcommand{\delz}{\frac{\partial}{\partial z}}

\newcommand{\xvec}{\hat{x}}
\newcommand{\yvec}{\hat{y}}
\newcommand{\zvec}{\hat{z}}
\newcommand{\rvec}[1]{\vec{r}_{#1}}

\usepackage{calligra}

\DeclareMathAlphabet{\mathcalligra}{T1}{calligra}{m}{n}
\DeclareFontShape{T1}{calligra}{m}{n}{<->s*[2.2]callig15}{}
\newcommand{\scriptr}{\mathcalligra{r}\,}
\newcommand{\boldscriptr}{\pmb{\mathcalligra{r}}\,}

\setlength{\parindent}{0pt}
\usepackage[margin=1.75in]{geometry}
\title{Griffiths Introduction to Electrodynamics 4th Edition Solutions to Selected Problems}
\author{}
\date{}

\begin{document}

\maketitle

\newpage

\subsection*{Problem 2.1} 

\paragraph{(a)} Twelve equal charges, $q$, are situated at the corners of a regular 12-sided polygon. What is the net force on a test charge $Q$ at the center?

\paragraph{Solution} The charges are arranged in a regular 12-sided polygon as depicted below. 

\input{tex_images/problem-2.1-polygon1}

If we form triangles by connecting each vertex to the center we see that the center angles are each $\theta = 2\pi / 12 = \pi/6$ radians.  Each vertex is an equal distance $r$ to the center, so the force that each charge exerts on a test charge $Q$ at the center is 

$$
F_E = \frac{1}{4\pi\varepsilon_0}\frac{qQ}{r^2}
$$

with the $x$ and $y$ components of the force given by 

\begin{align*}
    F_{E_x} &= F_E \cos(\theta) = \frac{1}{4\pi\varepsilon_0}\frac{qQ}{r^2} \cos(\theta) \\
    F_{E_y} &= F_E \sin(\theta) = \frac{1}{4\pi\varepsilon_0}\frac{qQ}{r^2} \sin(\theta)
\end{align*}

We can sum up all of the contributions for each $\theta$ by starting with $\theta = \frac{\pi}{2}$ and circling around counterclockwise by adding $k\frac{\pi}{6}$ for $k = 0...12$.  Doing this gives the following.

\begin{align*}
    \sum F_{E_x} &= \frac{1}{4\pi\varepsilon_0}\frac{qQ}{r^2} \sum_{k=0}
^{11} \cos(\frac{\pi}{2} + k\frac{\pi}{6})\\
\sum F_{E_y} &= \frac{1}{4\pi\varepsilon_0}\frac{qQ}{r^2} \sum_{k=0}
^{11} \sin(\frac{\pi}{2} + k\frac{\pi}{6})\\
\end{align*}

Instead of calculating these, we make use of symmetry. Notice that half of the terms for $\sum F_{E_x}$ will have negative values and half will have positive values. Since these pair off and are equal in magnitude, they zero out. The same argument holds for $\sum F_{E_y}$. Thus, the net force that a test charge $Q$ at the center feels is zero. 

\paragraph{(b)} Suppose one of the 12 $q$'s is removed (say the one at 6 o'clock). What is the force on $Q$? Explain your reasoning carefully.

\paragraph{Solution} If one of the charges is removed (the one at 6 o'clock say), then the forces no longer balance. From the equations above, we see that $F_{E_x}$ is still zero since the charges at 6 and 12 o'clock don't contribute to the x-direction of the force. However, when calculating $F_{E_y}$, there is a remaining charge exerting force 

$$
F_{E_y} = \frac{1}{4\pi\varepsilon_0}\frac{qQ}{r^2}
$$
on the test charge. If the charges have the same sign, then the force is repulsive. Otherwise it attractive.

\newpage

\paragraph{(c)} Now 13 equal charges $q$ are placed at the corners of a regular 13-sided polygon. What is the force on test charge $Q$ at the center?

\paragraph{Solution} The charge setup can be drawn as such. 

\input{tex_images/problem-2.1-polygon2}

We have the same basic equations as part (a), but we need to change the interior angle $theta$. The angle $\theta$ is now given by $\theta = \frac{2\pi}{13}$.  The components of the force in the $x$ and $y$ direction are now the following. 

\begin{align*}
    \sum F_{E_x} &= \frac{1}{4\pi\varepsilon_0}\frac{qQ}{r^2} \sum_{k=0}
^{12} \cos(\frac{\pi}{2} + k\frac{2\pi}{13})\\
\sum F_{E_y} &= \frac{1}{4\pi\varepsilon_0}\frac{qQ}{r^2} \sum_{k=0}
^{12} \sin(\frac{\pi}{2} + k\frac{2\pi}{13})\\
\end{align*}

The summations in each equation result in zero, so the net force the charge feels due to this arrangement is zero. 

\paragraph{(d)} If we remove the charge at a point, the net force is no longer zero.  Say we remove the $i^{th}$ charge, then the net force in the $x$ and $y$ direction is 

\begin{align*}
    F_{E_x} &= -\frac{1}{4\pi\varepsilon_0}\frac{qQ}{r^2}\cos(\frac{\pi}{2} + i\frac{2\pi}{13}) \\
F_{E_y} &= -\frac{1}{4\pi\varepsilon_0}\frac{qQ}{r^2}\sin(\frac{\pi}{2} + i\frac{2\pi}{13}) 
\end{align*}
where the negative signs signify the removal of the $i^{th}$ charge.

\newpage

\subsection*{Problem 2.2}
\paragraph{(a)} Find the electric field (magnitude and direction) a distance $z$ above the midpoint between two equal charges, $q$, a distance $d$ apart. Check that your result is consistent with what you expect when $z >> d$.

\paragraph{Solution} The setup of the problem is described in the picture below. 

\input{tex_images/problem-2.2a}

Let the angle between the origin, the charge, and point $z$ be denoted as $\theta$. Since the charges are equal and the same distance from the origin, we can immediately identify that the net electric field in the $x$ direction is zero due to symmetry. All that is left is to find the electric field in the $y$ direction given by

$$
E_{y} = E\sin\theta.
$$

The electric field due to one charge at point $P$ is given by 
$$
E = \frac{1}{4\pi\varepsilon_0}\frac{q}{z^2 + d^2/4}
$$
where $\sqrt{z^2 + d^2/4}$ is the radial distance from one of the charges to $P$ given by the Pythagorean theorem.  Additionally, we note that by the geometry of the problem we have 

$$
\sin\theta = \frac{z}{\sqrt{z^2 + d^2/4}}.
$$

Substituting these into the equation for $E_y$ gives and accounting for the field due to both charges gives

$$
E_{tot} = \frac{z}{2\pi\varepsilon_0}\frac{q}{(z^2 + d^2/4)^{3/2}}\cdot\vec{y} 
$$

In the limit, if we let $z$ grow much greater than $d$ the equation becomes the normal Coulomb's Law for a point charge but with twice the charge. 

\paragraph{(b)} Repeat part (a), only this time make the right-hand charge -q instead of +q.

\paragraph{Solution} If we allow the right-hand charge to be -q instead of +q, we hav a similar setup, but this time the $y$ components cancel and there is a net field in the $x$ direction. Indeed we see that the field due to the negative charge we have

\begin{align*}
    E_x = E\cos \theta &= \frac{1}{4\pi\varepsilon_0}\frac{q}{z^2 + d^2/4}\frac{d}{2\sqrt{z^2 + d^2/4}}\\
    &= \frac{d}{8\pi\varepsilon_0}\frac{q}{(z^2 + d^2/4)^{3/2}}
\end{align*}

And the total field at $P$ is then 
$$
E_{tot} = 2E_x = \frac{d}{4\pi\varepsilon_0}\frac{q}{(z^2 + d^2/4)^{3/2}}\cdot \vec{x}
$$

\newpage

\subsection*{Problem 2.3}
Find the electric field a distance $z$ above one end of a straight line segment of length $L$, which carries a uniform line charge $\lambda$. Check that your formula is consistent with what you would expect for the case $z >> L$.

\paragraph{Solution} The situation can be drawn as below. 

\input{tex_images/problem-2.3}

Since there is no symmetry in the problem, we expect to have to calculate both the $x$ and $y$ components of the electric field. The contribution of a small element $dl$ to the field at point $P$ is

$$
dE = \frac{1}{4\pi\varepsilon_0}\frac{dq}{r^2} = \frac{1}{4\pi\varepsilon_0}\cdot\frac{\lambda}{l^2 + z^2}dl.
$$

The $x$ and $y$ contributions are given respectively by 
\begin{align*}
    dE_x &= \frac{1}{4\pi\varepsilon_0}\frac{dq}{r^2} = \frac{1}{4\pi\varepsilon_0}\cdot\frac{\lambda}{l^2 + z^2}\cos\theta\ dl \\
    dE_y &= \frac{1}{4\pi\varepsilon_0}\frac{dq}{r^2} = \frac{1}{4\pi\varepsilon_0}\cdot\frac{\lambda}{l^2 + z^2}\sin\theta\ dl \\
\end{align*}

Substituting for $\sin\theta = z/(\sqrt{l^2 + z^2})$ and $\cos \theta = l/(l^2 + z^2)$ gives

\begin{align}
    dE_x &= \frac{l}{4\pi\varepsilon_0}\cdot\frac{\lambda}{(l^2 + z^2)^{3/2}}\ dl \\
    dE_y &= \frac{z}{4\pi\varepsilon_0}\cdot\frac{\lambda}{(l^2 + z^2)^{3/2}}\ dl
\end{align}

Integrating equation (1) from 0 to L gives 
\begin{align*}
    E_x &= \int_{0}^{L}dE_x = \int_{0}^{L} \frac{l}{4\pi\varepsilon_0}\cdot\frac{\lambda}{(l^2 + z^2)^{3/2}}\ dl \\
    &= \frac{\lambda}{4\pi\varepsilon_0}\int_{0}^{L}\frac{l}{(l^2 + z^2)^{3/2}}\ dl \\
    &= \frac{\lambda}{4\pi\varepsilon_0} \left[ -\frac{1}{\sqrt{l^2 + z^2}} \right]_{0}^L = \frac{\lambda}{4\pi\varepsilon_0}\left( \frac{1}{z} - \frac{1}{\sqrt{L^2 + z^2}}\right)
\end{align*}

Integrating equation (2) gives

\begin{align*}
    E_y &= \int_0^L dE_y = \frac{z\lambda}{4\pi\varepsilon_0}\int_0^L \frac{1}{(l^2 + z^2)^{3/2}}\ dl \\
    &= \frac{z\lambda}{4\pi\varepsilon_0} \left[ \frac{1}{z^2\sqrt{l^2 + z^2}}\right]^L_0\\
    &= \frac{\lambda}{4\pi\varepsilon_0} \left[ \frac{1}{z\sqrt{l^2 + z^2}}\right]^L_0 = \frac{\lambda}{4\pi\varepsilon_0} \left( \frac{1}{z\sqrt{L^2 + z^2}} - \frac{1}{z^2} \right)
\end{align*}

We can find the angle and magnitude that the force acts upon using trigonometry. 

\begin{align*}
    \tan\theta &= \frac{E_y}{E_x} = \frac{\frac{\lambda}{4\pi\varepsilon_0} \left( \frac{1}{z\sqrt{L^2 + z^2}} - \frac{1}{z^2} \right)}{\frac{\lambda}{4\pi\varepsilon_0}\left( \frac{1}{z} - \frac{1}{\sqrt{L^2 + z^2}}\right)} = \frac{-1}{z} \\
    \\
    \implies \theta &= \arctan(\frac{-1}{z})
\end{align*}

The magnitude of the electric field is then 

\begin{align*}
    E = \frac{\lambda}{4\pi\varepsilon_0}\left( \frac{1}{z} - \frac{1}{\sqrt{L^2 + z^2}}\right)\cos(\arctan(\frac{-1}{z}))
\end{align*}


\newpage

\subsection*{Problem 2.4} Find the electric field a distance z above the center of a square loop (side $a$) carrying uniform line charge $\lambda$.

\begin{figure}[h]
    \centering
    \includegraphics[width=0.5\linewidth]{images/square-loop.png}
\end{figure}

\paragraph{Solution} Since the square loop is symmetric, we only need to calculate the $y$ component of the electric field. We can use the result of example 2.2, which found the strength of the electric field a distance z above the center of a straight wire.  From example 2.2, the starting equation is 

\begin{align*}
    E = \frac{1}{4\pi\varepsilon_0}{\frac{2\lambda L}{z\sqrt{z^2  + L^2}}}
\end{align*}

In the example, the length of the wire is $2L$, and so we need to set $L = \frac{a}{2}$. Making this substitution gives the following.

\begin{align*}
    E = \frac{1}{4\pi\varepsilon_0}{\frac{\lambda a}{z\sqrt{z^2  + a^2/4}}}
\end{align*}

The $y$ component of the field is given by $E_y = E\sin\theta$ where $\theta = 2z/a$. So the, vertical electric field due to one side of the loop is 

\begin{align*}
    E_y = \frac{1}{2\pi\varepsilon_0}{\frac{\lambda}{\sqrt{z^2  + a^2/4}}}
\end{align*}

with the total electric field given by 

$$
 E_y = \frac{2}{\pi\varepsilon_0}{\frac{\lambda}{\sqrt{z^2  + a^2/4}}}.
$$

\newpage

\section*{Problem 2.5}
Find the electric field a distance $z$ above the center of a circular loop of radius $r$ that carries a uniform line charge $\lambda$.

\begin{figure}[h]
    \centering
    \includegraphics[width=0.75\linewidth]{images/circular-loop.png}
\end{figure}

\paragraph{Solution} The loop has radial symmetry, so we do not need to worry about calculating the $x$ component of the electric field. For electric field, a small arc $dl$ of the loop contributes an amount $dq = \lambda dl$ to the field given by 

\begin{align*}
    dE = \frac{1}{4\pi\varepsilon_0}\frac{\lambda}{r^2 + z^2}dl
\end{align*}

In the $y$ direction, we have 

\begin{align*}
    dE_y = \frac{1}{4\pi\varepsilon_0}\frac{\lambda}{r^2 + z^2}\frac{z}{\sqrt{r^2 + z^2}}dl = \frac{1}{4\pi\varepsilon_0}\frac{z\lambda}{(r^2 + z^2)^{3/2}}dl
\end{align*}

Integrating around the whole length of the loop gives 

\begin{align*}
    E_y &= \int_{0}^{2\pi r}\frac{1}{4\pi\varepsilon_0}\frac{z\lambda}{(r^2 + z^2)^{3/2}}dl = \frac{1}{4\pi\varepsilon_0}\frac{z\lambda}{(r^2 + z^2)^{3/2}} \int_{0}^{2\pi r} dl \\ 
    \\
    &= \frac{2\pi r}{4\pi\varepsilon_0}\frac{z\lambda}{(r^2 + z^2)^{3/2}} = \frac{r}{2\varepsilon_0}\frac{z\lambda}{(r^2 + z^2)^{3/2}}
\end{align*}

\newpage

\section*{Problem 2.6}

Find the electric field a distance z above the center of a flat circular disk of radius $R$ that carries a uniform surface charge $\sigma$. What does your formula give in the limit $R \leftarrow \infty$? Also check the case $z >> R$.

\begin{figure}[h]
    \centering
    \includegraphics[width=0.25\linewidth]{images/circular-disk.png}
\end{figure}

\paragraph{Solution} Since the disc has radial symmetry, we only need to calculate the field in the vertical direction.  A small ring of the disk contributes an amount $dq = \sigma r dr d\theta$ to electric field at point $P$. Writing this small contribution out we have the following.

\begin{align*}
dE_y = \frac{1}{4\pi\varepsilon_0}\frac{dq}{r^2 + z^2}\frac{z}{\sqrt{r^2 + z^2}} &= 
\frac{1}{4\pi\varepsilon_0}\frac{\sigma r \ dr \ d\theta}{r^2 + z^2}\frac{z}{\sqrt{r^2 + z^2}} = \frac{1}{4\pi\varepsilon_0}\frac{\sigma rz}{(r^2 + z^2)^{3/2}} \ dr \ d\theta
\end{align*}

Integrating this over $r$ and $\theta$ gives the following result.

\begin{align*}
    E_y &= \frac{\sigma z}{4\pi\varepsilon_0} \int_0^{2\pi} \int_0^R \frac{r}{(r^2 + z^2)^{3/2}} \ dr \ d\theta \\
    \\
    &= \frac{\sigma z}{4\pi\varepsilon_0} \int_0^{2\pi} \left[ \frac{1}{z} - \frac{1}{\sqrt{R^2 + z^2}}\right] \ d\theta \\
    \\
    &= \frac{2\pi\sigma z}{4\pi\varepsilon_0}\left[ \frac{1}{z} - \frac{1}{\sqrt{R^2 + z^2}}\right] = \frac{\sigma}{2\varepsilon_0}\left[ 1 - \frac{z}{\sqrt{R^2 + z^2}}\right]
\end{align*}

If we let $R \rightarrow \infty$, then the result becomes

\begin{align*}
    \lim_{R \rightarrow \infty}E_y = \frac{\sigma}{2\varepsilon_0}.
\end{align*}

On the other hand, if $z >> R$, then we have 

\begin{align*}
    \lim_{z \rightarrow \infty}E_y = 0.
\end{align*}

\newpage

\subsection*{Problem 2.7}
Find the electric field a distance z from the center of a spherical surface of radius $R$ that carries a uniform charge density $\sigma$. Treat the case $z < R $(inside) as well as $z > R$ (outside). Express your answers in terms of the total charge q on the sphere.

\begin{figure}[h]
    \centering
    \includegraphics[width=0.5\linewidth]{images/spherical-shell.png}
\end{figure}

\paragraph{Solution} For simplicity of notation, I denote $\scriptr$ as $r$. 
This problem can be simplified by using the law of cosines to rewrite $r$ as 
\begin{align*}
    r^2 = R^2 + z^2 - 2zR\cos\theta
\end{align*}

A small patch of area $dA$ contributes $dq = \sigma dA$ amount of charge to the electric field at point $P$. Using spherical coordinates, we can write $dA = R^2\sin \theta d\theta d\phi$, where $\phi$ is the angle of rotation on the $x-y$ plane. Thus, the contribution $dE$ to the electric field is 

\begin{align*}
    dE = \frac{1}{4\pi\varepsilon_0}\frac{R^2\sigma}{R^2 + z^2 - 2zR\cos\theta} \sin \theta d\theta d\phi
\end{align*}
Since the shell exhibits radial symmetry, we only need to calculate the field in the $z$ direction. If we allow the angle formed by $z$ and $r$ to be denoted by $\varphi$, then we can calculate the $z$ component using $\cos \varphi$. Using the law of cosines, substituting for r, and solving for $\cos \varphi$, we have 

\begin{align*}
    \cos \varphi = \frac{z - R\cos \theta}{\sqrt{R^2 + z^2 - 2zR\cos \theta}}.
\end{align*}

Multiplying $dE$ by $\cos \varphi$ gives the $z$ component of the field. 

\begin{align*}
    dE_z = \frac{R^2\sigma}{4\pi\varepsilon_0}\frac{z - R\cos \theta}{(R^2 + z^2 - 2zR\cos\theta)^{3/2}} \sin \theta d\theta d\phi
\end{align*}

Integrating over $\theta$ and $\phi$ gives 

\begin{align*}
    E_z &= \int_0^{\pi} \int_0^{2\pi} \frac{R^2\sigma}{4\pi\varepsilon_0}\frac{z - R\cos \theta}{(R^2 + z^2 - 2zR\cos\theta)^{3/2}} \sin \theta  d\phi d\theta \\
    E_z &= \frac{R^2\sigma}{4\pi\varepsilon_0} \int_0^{\pi} \int_0^{2\pi} \frac{z - R\cos \theta}{(R^2 + z^2 - 2zR\cos\theta)^{3/2}} \sin \theta  d\phi d\theta \\
    E_z&= \frac{2\pi R^2\sigma}{4\pi\varepsilon_0} \int_0^{\pi} \frac{z - R\cos \theta}{(R^2 + z^2 - 2zR\cos\theta)^{3/2}} \sin \theta d\theta
\end{align*}

The last integral can be solved by setting $u = \cos \theta$, splitting the fraction, and applying integration to each term. Doing this gives

\begin{align*}
    \int_0^{\pi} \frac{z - R\cos \theta}{(R^2 + z^2 - 2zR\cos\theta)^{3/2}} \sin \theta d\theta = \int_{1}^{-1} \left[ \frac{z}{(R^2 + z^2 - 2zRu)^{3/2}} - \frac{Ru}{(R^2 + z^2 - 2zRu)^{3/2}}\right] du
\end{align*}

The integral of the first term is 

\begin{align*}
    \int_{1}^{-1} \frac{z}{(R^2 + z^2 - 2zRu)^{3/2}}\ du &= \left[ \frac{1}{R\sqrt{R^2 + z^2 - 2zRu}} \right]^{-1}_{1}\\
    &=  \frac{1}{R\sqrt{R^2 + z^2 + 2zR}} - \frac{1}{R\sqrt{R^2 + z^2 - 2zR}} \\
    &= \frac{1}{R\sqrt{(R + z)^2}} - \frac{1}{R\sqrt{(R - z)^2}} \\
    &= \frac{z^2}{Rz^2|R + z|} - \frac{z^2}{Rz^2|R - z|} \\
\end{align*}

The integral of the second term is 

\begin{align*}
    \int_{1}^{-1}\frac{Ru}{(R^2 + z^2 - 2zRu)^{3/2}}\ du &= \left[ \frac{R^2 + z^2 - zRu}{Rz^2\sqrt{R^2 + z^2 - 2zRu}} \right]^{-1}_{1} \\
    &= \frac{R^2 + z^2 + zR}{Rz^2\sqrt{R^2 + z^2 + 2zR}} - \frac{R^2 + z^2 - zR}{Rz^2\sqrt{R^2 + z^2 - 2zR}} \\
    &= \frac{R^2 + z^2 + zR}{Rz^2\sqrt{(R + z)^2}} - \frac{R^2 + z^2 - zR}{Rz^2\sqrt{(R - z)^2}} \\
    &= \frac{R^2 + z^2 + zR}{Rz^2 |R + z|} - \frac{R^2 + z^2 - zR}{Rz^2|R - z|}
\end{align*}

The difference of these two expressions gives the full integral.

\begin{align*}
\int_0^{\pi} \frac{z - R\cos \theta}{(R^2 + z^2 - 2zR\cos\theta)^{3/2}} \sin \theta d\theta = \frac{1}{z^2}\left[ -\frac{R + z}{|z + R|} + \frac{R - z}{|R - z|} \right]
\end{align*}

So the full equation for $E_z$ is then 

\begin{align*}
    E_z = \frac{-2\pi R^2\sigma}{4z^2\pi\varepsilon_0}\left[ -\frac{R + z}{|R + z|} + \frac{R - z}{|R - z|} \right]
\end{align*}
where the negative sign on the the first term comes from the fact that we used u-substitution for the integral.\\

If $z > R$, then point $P$ is outside the sphere and our equation simplifies to

\begin{align*}
    E_z = \frac{\sigma}{\varepsilon_0}\left(\frac{R}{z}\right)^2
\end{align*}

If $z < R$ and so the point $P$ is on the inside of the sphere then the equation reduces to 
\begin{align*}
    E_z = 0
\end{align*}
since the terms in brackets cancel.

\newpage

\subsection*{Problem 2.8}
Use your result in Prob. 2.7 to find the field inside and outside a solid
sphere of radius $R$ that carries a uniform volume charge density $\rho$. Express your answers in terms of the total charge of the sphere, $q$. Draw a graph of $|E|$ as a function of the distance from the center.

\paragraph{Solution} To solve this, the result from problem 2.7 only needs to be integrated with respect to $r$. Replacing $R$ with $r$ and $\sigma$ with $\rho$ and integrating over the radius yields. 

\begin{align*}
    E_z = \int_0^R \frac{\rho}{\varepsilon_0}\left(\frac{r}{z}\right)^2 dr = \frac{\rho}{3\varepsilon_0}\frac{R^3}{z^2}
\end{align*}

\newpage

\subsection*{Problem 2.9}
Suppose the electric field in some region is found to be $E = kr^3\vec{r}$, in spherical coordinates ($k$ is some constant).

\paragraph{(a)} Find the charge density $\rho$.
\paragraph{Solution} To find the charge density, we use $Gauss$ equation and use spherical coordinates for the divergence..

\begin{align*}
    \nabla \cdot E &= \frac{\rho}{\varepsilon_0}\\
    \frac{1}{r^2}\frac{\partial }{\partial r}\left(r^2\cdot kr^3 \right) &= \frac{\rho}{\varepsilon_0}\\
    5kr^2 &= \frac{\rho}{\varepsilon_0}\\
    \rho &= 5k\varepsilon_0r^2
\end{align*}

\paragraph{(b)} Find the total charge contained in a sphere of radius $R$ centered at the origin. (Do it two different ways).

\paragraph{Solution} The first is to use a volume integral in spherical coordinates where we multiply the charge density for a small volume, denoted by $\rho(r, \theta, \phi) = 5k\varepsilon_0r^2$ by an infinitesimal volume element $d\tau$.

\begin{align*}
    Q &= \int_0^R \int_0^{\pi} \int_0^{2\pi} \rho(r, \theta, \phi) r^2\sin\theta d\phi d\theta dr \\
    &= \int_0^R \int_0^{\pi} \int_0^{2\pi} 5k\varepsilon_0 r^4\sin\theta d\phi d\theta dr \\
    &= 5k\varepsilon_0  \int_0^R \int_0^{\pi} \int_0^{2\pi} r^4\sin\theta d\phi d\theta dr \\
    &= 10\pi k \varepsilon_0 \int_0^{R} \int_0^{\pi}  r^4 \sin\theta d\theta dr \\
    &= 20\pi k \varepsilon_0 \int_0^{R}  r^4 d\theta dr\\
    &= 4\pi k \varepsilon_0 r^5
\end{align*}

The second way is to divide the sphere into thin concentric shells and integrate from $0$ to $R$. A thin shell of radius $r$ has total charge $4(5k\varepsilon_0r^2) \pi r^2\cdot dr = 20k\pi\varepsilon_0r^4\cdot dr$. Integrating this from 0 to $R$ gives the same result. 

\newpage

\subsection*{Problem 2.10} 
A charge $q$ sits at the back corner of a cube. What is the flux of $E$ through the shaded side?

\begin{figure}[h]
    \centering
    \includegraphics[width=0.5\linewidth]{images/flux-through-cube.png}
\end{figure}

\paragraph{Solution} We cannot use Gauss's law outright here since the charge straddles several faces of the cube. If it were in the center of the cube, then the flux through one side would simply be one-sixth of the entire flux. However, we can construct a cube with the charge at the center by adding three more cubes that meet at the charge.\\

We now have a cube with sides 4 times larger than the original, and there is electric flux through every side. Since there are 6 sides, the flux through one side is 
$$
\Phi_E^{big \ cube} = \frac{q}{6\varepsilon_0}
$$

And since one side of the big cube is 4 times larger than that of the small cube, the flux through the original surface is 

$$
\Phi_E = \frac{q}{24\varepsilon_0}
$$

\newpage

\subsection*{Problem 2.11}
Use Gauss's law to find the electric field inside and outside a spherical shell of radius $R$, which carries a uniform surface charge $\sigma$.

\paragraph{Solution} The shell has radius $R$ and unit surface charge per unit area of $\sigma$. Let $z$ be the distance from the center of the shell. Gauss's law says 

$$
\oint_{\mathcal{S}} E \cdot d\vec{a} = \frac{Q_{enc}}{\varepsilon_0} 
$$

Since $E$ is always parallel to $d\vec{a}$ and is constant for a given radius, we can write 

$$
E \oint_{\mathcal{S}}d\vec{a} = \frac{Q_{enc}}{\varepsilon_0}.
$$

If $z < R$, then we are inside the shell and enclose no charge, thus the right hand side is zero and the equation reduces to $E = 0$.  If $z > R$, then we have 

\begin{align*}
    E\oint_{\mathcal{S}}d\vec{a} &= \frac{Q_{enc}}{\varepsilon_0} \\
    E 4\pi z^2 &= \frac{\sigma 4\pi R^2}{\varepsilon_0}\\
    E &= \frac{\sigma}{\varepsilon_0}\frac{R^2}{z^2}
\end{align*}

The result is the same as problem 2.7.

\newpage
\subsection*{Problem 2.12}
Use Gauss's law to find the electric field inside a uniformly charged sphere.

\paragraph{Solution} We use the same approach as problem 2.11. Let $z$ be the distance from the center of the sphere.  We have to more complex cases here. \\

If $z < R$, then the total charge enclosed at a distance $z$ from the center is 
$$
Q_{enc} = \rho \frac{4\pi z^3}{3}
$$ 

and so we have 

$$
E =  \frac{1}{4\pi z^2}\frac{\rho 4\pi z^3}{3\varepsilon_0} = \frac{\rho z}{3\varepsilon_0}
$$

If $z > R$, then the enclosed charge is 

$$Q_{enc} = \rho \frac{4\pi R^3}{3\varepsilon_0}$$

and the electric field is 

$$
E = \frac{1}{4\pi z^2}\frac{\rho 4\pi R^3}{3\varepsilon_0} = \frac{\rho R^3}{3z^2\varepsilon_0}
$$

\newpage

\subsection*{2.13}
Find the electric field a distance $s$ from an infinitely long straight wire that carries a uniform line charge $\lambda$.

\paragraph{Solution} Gauss's law says that 

$$
\oint_{\mathcal{S}}E \cdot d\vec{a} = \frac{Q_{enc}}{\varepsilon_0}
$$

Since the electric field lines are already in the same direction of $d\vec{a}$, we can rewrite this as 

$$
\oint_{\mathcal{S}}E\ d\vec{a} = \frac{Q_{enc}}{\varepsilon_0}.
$$

By symmetry, we expect the field to be uniform at each unique $s$, so for sufficiently long wire $l$ we have 

\begin{align*}
    E(2\pi s l) &= \frac{\lambda l}{\varepsilon} \\
    E &= \frac{\lambda}{2\pi s \varepsilon_0}
\end{align*}

\newpage

\subsection*{Problem 2.14} 
Find the electric field inside a sphere that carries a charge density proportional to the distance from the origin, $\rho = kr$, for some constant $k$.

\paragraph{Solution} Let $z$ denote the distance from the center of the sphere. The total charge enclosed a distance $z$ within the sphere is 

\begin{align*}
    Q_{enc} = \int_0^z kr4\pi r^2 dr = 4\pi k \int_0^z r^3 dr = \pi k z^4.
\end{align*}

Gauss's law then says 

$$
\oint_{\mathcal{S}}E \cdot d\vec{a} = \frac{1}{\varepsilon_0}\pi k z^4
$$

By symmetry, we expect the electric field strength to be the same for the same distance $z$, so 

\begin{align*}
    E(4\pi z^2) &= \frac{1}{\varepsilon_0}\pi k z^4 \\
    E &= \frac{1}{4\varepsilon_0}k z^2
\end{align*}


\newpage
\subsection*{Problem 2.15}
A thick spherical shell carries charge density
$$
\rho = \frac{k}{r^2}
$$
in the region $a \leq r \leq b$. Find the electric field in three regions: (i) $r < a$, (ii) $a < r< b$, (iii) $r > b$. Plot $|E|$ as a function of $r$.

\paragraph{Solution} For the first case where $r < a$, there is no enclosed charge and so $E = 0$ everywhere. \\

For the second case where $a < r< b$, the enclosed charge is given by

\begin{align*}
    Q_{enc} &= \int_a^r \rho 4\pi s^2 ds
    = k4\pi \int_a^r ds = 4k\pi(r - a)
\end{align*}

and the electric field in this region is 

\begin{align*}
    E (4\pi r^2) &= 4k\pi(r - a) \\
    E &= \frac{k(r - a)}{\varepsilon_0} \frac{1}{r^2}.
\end{align*}
For the third case, the enclosed charge is 

\begin{align*}
    Q_{enc} = \int_a^b \frac{k}{r^2}4\pi r^2dr = 4\pi k \int_a^b dr = 4\pi k(b - a)
\end{align*}

Then Gauss's law states 

\begin{align*}
    E(4\pi r^2) &= \frac{4\pi k (b - a)}{\varepsilon_0} \\
    E &= \frac{k(b - a)}{\varepsilon_0}\frac{1}{r^2}
\end{align*}

\newpage

\subsection*{Problem 2.16}
 A long coaxial cable carries a uniform volume charge density $\rho$ on the inner cylinder (radius $a$), and a uniform surface charge density on the outer cylindrical shell (radius $b$). This surface charge is negative and is of just the right magnitude such that the cable as a whole is electrically neutral. 

Find the electric field in each of the three regions:
\begin{enumerate}
    \item Inside the inner cylinder ($s < a$),
    \item Between the cylinders ($a < s < b$),
    \item Outside the cable ($s > b$).
\end{enumerate}

Plot $|E|$ as a function of $s$.

\paragraph{Solution}  In the first case, we only need to calculate the electric field due to the inner cylinder since the electric field from the outer shell cancels inside of the conductor.  For $s < a$, and length $l$ of the cable, the total enclosed charge is 

$$
Q_{enc} = \rho \pi r^2 l
$$

Gauss's law then gives 

$$
E = \frac{1}{2\pi r l}\frac{\rho \pi r^2 l}{\varepsilon_0} = \frac{\rho r}{2\varepsilon_0}
$$

For the second case, the field due to the outer shell still contributes nothing, so for $ a< r< b$, the total enclosed charge is $Q_{enc} = \rho\pi a^2 l$. We use our result in part (i) to obtain the following.

$$
E = \frac{1}{2\pi r l}\frac{\rho \pi a^2 l}{\varepsilon_0} = \frac{\rho}{2\varepsilon_0}\frac{a^2}{r}
$$


For the third case, the total enclosed charge is 0 since the problem states that the entire coaxial cable is electrically neutral. Thus, for Gauss's law to hold it must be the case that $E = 0$ everywhere outside the cable. 


\newpage

\subsection*{Problem 2.17}

An infinite plane slab, of thickness 2d, carries a uniform volume charge density $\rho$. Find the electric field, as a function of $y$, where $y = 0$ at the center. Plot $ E$ versus $y$, calling $E$ positive when it points in the $+y$ direction and negative when it points in the $-y$ direction.

\begin{figure}[h]
    \centering
    \includegraphics[width=0.35\linewidth]{images/large-slab.png}
\end{figure}

\paragraph{Solution} The electric fields point directly outward on both sides of the slab. For an arbitrary area $A$, the enclosed charge is 

$$
Q_{enc} = 2|y|A\rho
$$

And Gauss's law states

\begin{align*}
E (2A) &= \frac{2|y|A\rho}{\varepsilon} \\
E &= \frac{|y|\rho}{\varepsilon_0}
\end{align*}

If $y > d$, then the equation becomes

$$
E = \frac{d\rho}{\varepsilon_0}.
$$

\newpage

\subsection*{Problem 2.18}
 Two spheres, each of radius $R$ and carrying uniform volume charge densities $+\rho$ and $-\rho$, respectively, are placed so that they partially overlap. Call the vector from the positive center to the negative center $\vec{d}$.  Show
 that the field in the region of overlap is constant, and find its value.

 \begin{figure}[h]
     \centering
     \includegraphics[width=0.5\linewidth]{images/overlapping-spheres.png}
 \end{figure}

 \paragraph{Solution} We use the solution from problem 2.12 to solve for the electric field in the overlapping region.  The electric field a radial distance $r$ from the center of a sphere with a unit-volume charge $\rho$ is given by 
 
 $$
 E = \frac{\rho}{3\varepsilon_0}r.
 $$

 We note that the radial distance from the center of each sphere to the region of overlap is $d - R$. The length of the region of overlap is given by $2R - d$. Denoting the length of $\vec{d}$ by d, the electric field in the region of overlap due to the positive sphere and negative sphere are, respectively, 

 \begin{align*}
     E^+ &= \frac{\rho}{3\varepsilon_0}(d - R + z)\vec{d} \quad \text{ for } 0\leq z \leq 2R- d \\
     E^- &= \frac{\rho}{3\varepsilon_0}(R - z)\vec{d} \quad \text{ for } 0\leq z \leq 2R- d
 \end{align*}

Since the fields are constructive, we add their magnitudes with the understanding that the fields point in the direction of $\vec{d}$.

\begin{align*}
    E = E^+ + E^- = \frac{\rho}{3\varepsilon_0}(d - R + z) + \frac{\rho}{3\varepsilon_0}(R - z) = \frac{\rho}{3\varepsilon_0}d \vec{d}
\end{align*}

The equation for $E$ depends only on the distance between the center of the two spheres, and so the field is constant in the region of overlap.

\newpage

\subsection*{Problem 2.19}
 Calculate $\nabla \times E$ directly from Eq. 2.8, by the method of Sect. 2.2.2.  Refer to Prob. 1.63 if you get stuck. 
 
\begin{align*}
    E(\mathbf{r}) = \frac{1}{4\pi \varepsilon_0}\int \frac{\rho(\mathbf{r}')}{r^2}\hat{r}\ d\tau'
\end{align*}

\paragraph{Solution} By the method in Sect. 2.2.2, we take the curl of both sides. 

$$
\nabla \times E(\mathbf{r}) = \nabla \times \left[\frac{1}{4\pi \varepsilon_0}\int \frac{\rho(\mathbf{r}')}{r^2}\hat{r}\ d\tau'\right]
$$

If we assume that the field is sufficiently smooth and continuously differentiable, then we can move the curl operator into the integral.

$$
\nabla \times E(\mathbf{r}) = \frac{1}{4\pi \varepsilon_0}\int \nabla \times \left ( \frac{\hat{r}}{r^2}\right)\rho(\mathbf{r}')\ d\tau'
$$
Since the electric field depends only on the distance from the collection of charges and points radially outward, it follows that $\nabla \times \frac{\hat{r}}{r^2} = 0$. For both sides of the equation to be true at all points in space, it must be the case that 

$$
\nabla \times E(\mathbf{r}) = 0.
$$

\newpage

\subsection*{Problem 2.20}
One of these is an impossible electrostatic field. Which one?

\paragraph{(a)} $E = k[xy\mathbf{\hat{x}}+2yz\mathbf{\hat{y}}+3xz\mathbf{\hat{z}}]$
\paragraph{(b)} $E = k[y^2\bvec{x} +(2xy+z^2)\bvec{y}+2yz\bvec{z}]$

\paragraph{Solution} In order for $E$ to be an electrostatic field, its curl must be zero. The curl for the field from (a), denoted $E_a$, is found to be

\begin{align*}
    \nabla \times E_a = \left | \begin{array}{ccc}
         \bvec{x}& \bvec{y} & \bvec{z}  \\
         \frac{\partial}{\partial x} & \frac{\partial}{\partial y} & \frac{\partial}{\partial z}\\ 
         xy & 2yz & 3xz
    \end{array} \right | &= \left(\frac{\partial}{\partial y}3xz - \frac{\partial}{\partial z}2yz\right)\bvec{x} - \left(\frac{\partial}{\partial x}3xz - \frac{\partial}{\partial z}xy\right)\bvec{y} + \left(\frac{\partial }{\partial x}2yz - \frac{\partial}{\partial y}xy\right)\bvec{z} \\
    &= -2y\bvec{x} - 3z\bvec{y} - x\bvec{z}
\end{align*}

For the field associated with (b), we calculate the curl in the same way.

\begin{align*}
    \nabla \times E_b = \left | \begin{array}{ccc}
         \bvec{x}& \bvec{y} & \bvec{z}  \\
         \frac{\partial}{\partial x} & \frac{\partial}{\partial y} & \frac{\partial}{\partial z}\\ 
         y^2 & 2xy + z^2 & 2yz
    \end{array} \right | &= \left ( \dely 2yz - \delz (2xy + z^2) \right )\bvec{x} - \left( \delx 2yz - \delz y^2 \right)\bvec{y} + \left(\delx 2yz - \dely xy \right)\bvec{z}\\
    &= 0\bvec{x} + 0\bvec{y} + 0\bvec{z}
\end{align*}

The second field is an electrostatic field since its curl is everywhere zero.

\newpage
\subsection*{Problem 2.21}
Find the potential inside and outside a uniformly charged solid sphere whose radius is $R$ and whose total charge is $q$. Use infinity as your reference point. Compute the gradient of V in each region, and check that it yields the correct field.

\paragraph{Solution} Outside the sphere, the electric field is given by Gauss's law as 

$$
E(r) = \frac{q}{4\pi\varepsilon_0}\frac{1}{r^2}
$$

with the potential equal to 

$$
V(r) = -\frac{q}{4\pi\varepsilon_0}\int_{\infty}^{r} \frac{1}{z^2} \ dz = \frac{q}{4\pi\varepsilon_0}\frac{1}{r}.
$$

Inside the sphere, we have that the field is 

$$
E(r) = \frac{q}{4\pi\varepsilon_0}\frac{r}{R^3}.
$$

We integrate from infinity to our point, $r$, inside the sphere to get the total potential. 

\begin{align*}
    V(r) &= -\int_{\infty}^R \frac{q}{4\pi\varepsilon_0}\frac{1}{z^2}\ dz -\int_{R}^r \frac{q}{4\pi\varepsilon_0}\frac{z}{R^3} \ dz \\
    \\
    &= \frac{q}{4\pi \varepsilon_0}\frac{1}{R} - \frac{q}{4\pi\varepsilon_0R^3}\left[ \frac{z^2}{2} \right]^r_R \\
    \\
    &= \frac{q}{4\pi \varepsilon_0R} - \frac{qr^2}{8\pi\varepsilon_0R^3} + \frac{q}{8\pi\varepsilon_0R}
\end{align*} 

\newpage
\subsection*{Problem 2.22}
Find the potential a distance $s$ from an infinitely long straight wire that carries a uniform line charge $\lambda$.

\paragraph{Solution} We create a Gaussian surface in the form of a cylinder centered on the wire. The surface will have length $l$ and radius $s$. The total enclosed charge is 

$$
Q_{enc} = \lambda l
$$

The enclosing surface has area $2\pi sl$, so our field a distance $s$ from the wire is 

$$
E = \frac{\lambda}{2s\pi \varepsilon_0} = \frac{\lambda}{2\pi\varepsilon_0}\frac{1}{s}.
$$

We set $s_0$ to be some reference point. The potential is then 

$$
V(s) = -\frac{\lambda}{2\pi s \varepsilon_0} \int_{s_0}^s \frac{1}{r}\ dr = -\frac{\lambda}{2\pi s \varepsilon_0} \ln(\dfrac{s}{s_0}).
$$

\newpage
\subsection*{Problem 2.23}
For the charge configuration of Prob. 2.15, find the potential at the
center, using infinity as your reference point.

\paragraph{Solution} The charge configuration for problem 2.15 involves a thick spherical shell with charge density $\rho = \dfrac{k}{r^2}$ in the region $a \leq r \leq b$. Outside the sphere, the electric field is given by 

\begin{align*}
    E(r) = \frac{k(b - a)}{\varepsilon_0} \frac{1}{r^2}
\end{align*}

and from $a \leq r \leq b$ the electric field is given by 

\begin{align*}
    E(r) = \frac{k}{\varepsilon_0}\frac{r - a}{r^2}
\end{align*}

The potential at the center is given by integrating the potential from infinity to 0.

\begin{align*}
    V(0) &= -\frac{k(b - a)}{\varepsilon_0} \int_{\infty}^b  \frac{1}{r^2}\ dr - \frac{k}{\varepsilon_0}\int_b^a \frac{r - a}{r^2}\ dr - \int_a^0 0 \ dr \\
    \\
    &= \frac{k(b - a)}{\varepsilon_0}\frac{1}{b} + \frac{\left(\ln\left(b\right) + \frac{a}{b} - \ln\left(a\right) - 1\right) k}{{\varepsilon}_{0}} \\
    &= \frac{k}{\varepsilon_0}\left[ \frac{b - a}{b} + \ln\left(b\right) + \frac{a}{b} - \ln\left(a\right) - 1\right]
\end{align*}

\newpage

\subsection*{Problem 2.24}
For the configuration of Prob. 2.16, find the potential difference between a point on the axis and a point on the outer cylinder.

\paragraph{Solution} In problem 2.16, we found that the electric field in the region from $0 \leq r \leq a$ was 

$$
E(r) = \frac{1}{2}\rho r
$$

and so the potential difference in this region is given by 

$$
V(r) = -\int_0^r\frac{1}{2}\rho s \ ds = -\frac{1}{4}\rho r^2
$$.

For the region where $a < r$, the electric field is 

$$
E(r) = \frac{\rho a^2}{2}\frac{1}{r}
$$

The potential in this region is given by 


\begin{align*}
    V(r) &= -\int_0^a\frac{1}{2}\rho s \ ds - \int_a^r \frac{\rho a^2}{2}\frac{1}{s}\ ds \\
    \\
    &= -\frac{1}{4}\rho a^2 - \frac{\rho a^2}{2}\ln(\frac{r}{a})
\end{align*}

\newpage

\subsection*{Problem 2.25}
Using Eqs.~(2.27) and (2.30), find the potential at a distance \( z \) above the
center of the charge distributions in Fig.~2.34. In each case, compute \( \mathbf{E} = -\nabla V \), and
compare your answers with Ex.~2.1, Ex.~2.2, and Prob.~2.6, respectively. Suppose
that we changed the right-hand charge in Fig.~2.34a to \( -q \); what then is the potential
at \( P \)? What field does that suggest? Compare your answer to Prob.~2.2, and explain
carefully any discrepancy.

\paragraph{Solution}
For the first scenario, we have two point charges arranged as in the diagram below. 

\begin{figure}[h]
    \centering
    \includegraphics[width=0.50\linewidth]{images/two-point-charges.png}
\end{figure}

The potential at point $P$ is given by 

\begin{align*}
    V_P = \frac{1}{4\pi \varepsilon_0} \frac{2q}{\sqrt{z^2 + \dfrac{d^2}{4}}}.
\end{align*}

To find the electric field, we only need to take the derivative with respect to $z$ since symmetry guarantees that the the field in the $x$ and $y$ directions cancel. 

\begin{align*}
    \delz\left[ \frac{1}{4\pi \varepsilon_0} \frac{2q}{\sqrt{z^2 + \dfrac{d^2}{4}}} \right] &= \frac{2q}{4\pi \varepsilon_0} \delz \left[\frac{1}{\sqrt{z^2 + \dfrac{d^2}{4}}} \right] \\
    \\
    &= -\frac{2q}{4\pi \varepsilon_0} \frac{z}{\left(z^2 + \frac{d^2}{4}\right)^{3/2}} \\
    \\
    &= -\frac{q}{\pi \varepsilon_0}\frac{z}{{\left(z^2 + \frac{d^2}{4}\right)^{3/2}}}
\end{align*}

So the electric field in the $z$ direction is given by 

$$
E_z =  \frac{q}{2\pi \varepsilon_0}\frac{z}{{\left(z^2 + \frac{d^2}{4}\right)^{3/2}}}
$$

In the second scenario we have a uniform and continous line charge as demonstrated in the diagram. 

\begin{figure}[h]
    \centering
    \includegraphics[width=0.35\linewidth]{images/line-charge.png}
\end{figure}

The potential is given by 

$$V_P = \frac{1}{4\pi \varepsilon_0} \int \frac{dq}{r}$$

where $dq = \lambda \ dl$ and $r = \sqrt{l^2 + z^2}$. Making this substitution and integrating gives 

\begin{align*}
    V_P = \frac{\lambda}{4\pi\varepsilon_0} \int_{-L}^L \frac{1}{\sqrt{l^2 + z^2}}\ dl = \frac{\lambda}{4\pi\varepsilon_0} \ln \left[ \frac{L + \sqrt{L^2 + z^2}}{-L + \sqrt{L^2 + z^2}}\right]
\end{align*}

To find the electric field from the potential, we can apply the partial derivative with respect to $z$ before integrating to simplify the calculation. 

\begin{align*}
    E_z &= -\delz \left[ \frac{\lambda}{4\pi\varepsilon_0} \int_{-L}^{L} \frac{1}{\sqrt{l^2 + ^2}}\ dl \right] \\
    &=-\frac{\lambda}{4\pi\varepsilon_0}\int_{-L}^L \delz \frac{1}{\sqrt{l^2 + z^2}}\ dl \\
    &= \frac{\lambda}{4\pi\varepsilon_0}\int_{-L}^L \frac{z}{(l^2 + z^2)^{3/2}}\ dl \\
    &= \frac{\lambda}{4\pi\varepsilon_0}\frac{2L}{z \sqrt{z^{2} + L^{2}}}
\end{align*}

In the third scenario we have a uniform surface charge distribution on a circular disc. 

\begin{figure}[h]
    \centering
    \includegraphics[width=0.5\linewidth]{images/charged-disc.png}
\end{figure}

The potential at point $P$ is given by 

$$
V_P = \frac{1}{4\pi\varepsilon_0} \int_0^R \frac{dq}{d}
$$

where $dq = \sigma 2\pi r\ dr$ and $d = \sqrt{r^2 + z^2}$. Making this substitution and integrating gives 
\begin{align*}
    V_P = \frac{2\pi\sigma}{4\pi \varepsilon_0}\int_0^R \frac{r}{\sqrt{r^2 + z^2}}\ dr = \frac{\sigma}{2 \varepsilon_0} \left[\sqrt{z^{2} + R^{2}} - z\right].
\end{align*}

The electric field for the $z$ component is 

\begin{align*}
    E_z &= -\delz \left[ \frac{\sigma}{2 \varepsilon_0} \left[\sqrt{z^{2} + R^{2}} - z\right] \right] \\
    &=-\frac{\sigma}{2\varepsilon_0}\left[ \frac{z}{\sqrt{z^2 + R^2}} - 1\right] \\
    &= \frac{\sigma}{2\varepsilon_0}\left[1 - \frac{z}{\sqrt{z^2 + R^2}}\right]
\end{align*}

\newpage

\subsection*{Problem 2.26} 
A conical surface (an empty ice cream cone) carries a uniform surface charge of $\sigma$. The heigh of the cone is $h$, as is the radius of the top. Find the potential difference between points $a$ (the vertex) and $b$ (the center of the top).

\paragraph{Solution}
We start with a diagram depicting the cone as a triangle with the top having radius h and the triangle as a whole having height h. The small triangle to the right depicts the change in slant height that arises from a small change $dz$ along the vertical axis. 

\input{tex_images/ice-cream-cone}

We compute the potential at point $a$ by dividing the cone into concentric frustrums and integrating with respect to $z$.  The potential is given by 

$$
V(a) = \frac{1}{4\pi \varepsilon_0} \int \frac{dq}{r}
$$

We first note that, by similarity of the triangles, we have 

$$
\frac{x}{h} = \frac{z}{h} \quad \implies x = z \quad \implies dx = dz
$$

A small increment of the surface area contributes $dq = \sigma\ dA$ to the potential at a. A circular ring with radius $x = z$ will carve out a small area along $ds$ equal to $dA = 2\pi z \ ds$.  From the smaller diagram, we see that $ds = \sqrt{(dz)^2 + (dz)^2} = \sqrt{2}\ dz$.  Also, from the diagram we have $r = \sqrt{z^2 + x^2} = \sqrt{2} z$. Then we have

\begin{align*}
    V(a) &= \frac{1}{4\pi \varepsilon_0} \int_0^h \frac{\sigma 2\sqrt{2}\pi z }{\sqrt{2}z}\ dz = \frac{\sigma}{2\varepsilon_0}h.
\end{align*}

The potential at point $b$ is more difficult to compute. In this computation, only the distance $r$ changes. The other leg of the triangle is given by $h - z$, so that the equation for $r$ is $r = \sqrt{z^2 + (h - z)^2}$. Thus, we need to integrate (using an integral calculator) the following.

\begin{align*}
    V(b) &= \frac{1}{4\pi \varepsilon_0} \int_0^h \frac{\sigma 2\sqrt{2}\pi z}{\sqrt{z^2 + (h - z)^2}}\ dz = \frac{\sigma \sqrt{2}}{2\varepsilon_0} \int_0^z \frac{z}{\sqrt{z^2 + (h - z)^2}}\ dz \\
    \\
    &= \frac{\sigma \sqrt{2}}{2\varepsilon_0}\left[\frac{h \ln\left(h \sqrt{\left(2z - h\right)^{2} + h^{2}} + h \left(2z - h\right)\right)}{2^{\frac{3}{2}}} + \frac{\sqrt{\left(z - h\right)^{2} + z^{2}}}{2}\right]^h_0 \\
    &= \frac{\sigma \sqrt{2}}{2\varepsilon_0}\frac{h}{2\sqrt{2}} \log \left( \frac{\sqrt{2} + 1}{\sqrt{2} - 1} \right) = \frac{\sigma}{4\varepsilon_0}h \log \left( \frac{\sqrt{2} + 1}{\sqrt{2} - 1} \right)
\end{align*}

So the potential difference between the two points is 

$$
V(b) - V(a) = \frac{\sigma}{2\varepsilon_0}h - \frac{\sigma}{4\varepsilon_0}h \log \left( \frac{\sqrt{2} + 1}{\sqrt{2} - 1} \right)
$$

\newpage

\subsection*{Problem 2.27}
Find the potential on the axis of a uniformly charged solid cylinder a distance $z$ from the center. The length of the cylinder is $L$, its radius is $R$, and the charge density is $\rho$. Use your result to calculate the electric field at this point.

\paragraph{Solution} We draw a diagram of the scenario below. Since $z$ is defined to be the distance from the center of the cylinder, the distance that $z$ sits from the top (or bottom) of the cylinder is $z - L/2$.

\input{tex_images/solid-cylinder}

The cylinder is solid, so we must use the potential due to a volume, which will require three integrals, but we can easily reduce it to two.  The general form of the potential is

$$
V(z) = \frac{1}{4\pi \varepsilon_0} \int \frac{dq}{r}.
$$

We imagine a thin 2d disc of radius $r'$ centered on the axis. For this given radius, a small element of volume $dV$ is given by $(2\pi r'\ dr)\cdot dh$ where $dh$ is an infinitely small change in height, effectively carving out a miniature cylinder with arbitrary thickness. The amount of charge contained in this sliver of volume is $dq = \rho dV = \rho (2\pi r'\ dr)\ dh$. At this specific radius $r'$, the distance between that point and point $P$ is $r = \sqrt{(r')^2 + (z - h)^2}$.  Intgrating with respect to the radius and the height gives the following integral set up. 

\begin{align*}
    V_P &= \frac{1}{4\pi \varepsilon_0} \int_{-L/2}^{L/2} \int_0^R \frac{2\pi \rho r'}{\sqrt{(r')^2 + (z - h)^2}} \ dr' dh = \frac{\rho}{2 \varepsilon_0} \int_{-L/2}^{L/2} \int_0^R \frac{r'}{\sqrt{(r')^2 + (z - h)^2}} \ dr' dh \\
    &= \frac{\rho}{2 \varepsilon_0}\int_{-L/2}^{L/2}\left[\sqrt{z^{2} - 2hz + h^{2} + R^{2}} - z + h\right]\ dh \\
    &= \left[\frac{(h - z)}{2} \sqrt{(h - z)^2 + R^2} + \frac{R^2}{2} \ln \left( h - z + \sqrt{(h - z)^2 + R^2} \right) -zh + \frac{h^2}{2}\right]_{-L/2}^{L/2} \\
    &=
- L z - \frac{R^{2}}{2} \ln \left( - \frac{L}{2} - z + \frac{\sqrt{4 R^{2} + (L + 2z)^{2}}}{2} \right) 
+ \frac{R^{2}}{2} \ln \left( \frac{L}{2} - z + \frac{\sqrt{4 R^{2} + (L - 2z)^{2}}}{2} \right) \\
&\quad + \frac{(L - 2z) \sqrt{4 R^{2} + (L - 2z)^{2}}}{8} 
+ \frac{(L + 2z) \sqrt{4 R^{2} + (L + 2z)^{2}}}{8}
\end{align*}

\newpage

\subsection*{Problem 2.29}
Check that Eq. 2.29 satisfies Poisson’s equation, by applying the Laplacian and using Eq. 1.102.

\paragraph{Solution} We must take the Laplacian of 

\begin{align*}
    V(\vec{r}) = \frac{1}{4\pi \varepsilon_0} \int \frac{\rho(\vec{r'})}{\scriptr} d\tau'
\end{align*}

Taking the Laplacian gives 

\begin{align*}
    \nabla^2 V(\vec{r}) &= \nabla^2 \left[\frac{1}{4\pi \varepsilon_0}\int \frac{\rho(\vec{r'})}{\scriptr} d\tau '\right] 
    = \frac{1}{4\pi \varepsilon_0}\int \nabla^2\left(\frac{\rho(\vec{r'})}{\scriptr}\right)d\tau
    = \frac{1}{4\pi \varepsilon_0}\int \rho(\vec{r'}) \nabla^2\left(\frac{1}{\scriptr}\right)d\tau
\end{align*}

Equation $1.102$ gives the Laplacian of $1/\scriptr$.

\begin{align*}
    \nabla^2(\dfrac{1}{\scriptr}) = -4\pi \delta^3(\scriptr)
\end{align*}

Substituting this gives 

\begin{align*}
    \nabla^2 V(\vec{r}) &= -\frac{1}{4\pi\varepsilon_0} \int \rho(\vec{r'})4\pi \delta^3(\scriptr)d \tau ' = -\frac{1}{\varepsilon_0} \int \rho(\vec{r'})\delta^3(\vec{r} - \vec{r'}) d\tau '
\end{align*}

The delta function is now integrated over the entire region, and has the property that at each $\vec{r} = \vec{r'}$, the function picks out the value of $\rho(\vec{r'})$ at that infinitesimally small unit of volume. Thus, the integral is just the charge density at $\vec{r}$.  So the equation reduces to 

$$
\nabla^2 V(\vec{r}) = -\frac{\rho(\vec{r})}{\varepsilon_0}
$$

\newpage

\subsection*{Problem 2.30}
\paragraph{(a)} Check that the results of Exs. 2.5 and 2.6, and Prob. 2.11, are consistent with Eq. 2.33.

\paragraph{Solution} To show that example 2.5 satisfies equation 2.33, we use the formula derived for the electric field due to an infinite sheet, given by $E_{\perp} = \frac{\sigma}{2\varepsilon_0}$. The fields just above and just below the surface of the sheet are equal in magnitude but opposite in direction. Taking direction into account we have 

$$
E_{above} - E_{below} = \frac{\sigma}{2\varepsilon_0} - \left(-\frac{\sigma}{2\varepsilon_0}\right) = \frac{\sigma}{\varepsilon_0}.
$$

For example 2.6, we were given that the field in between two oppositely charged, parallel plates is $\sigma/\varepsilon_0$ and outside the region between the plates the field is zero. Picking one of the plates, we have 

\begin{align*}
    E_{above} - E_{below} = \frac{\sigma}{\varepsilon_0} - (0)  = \frac{\sigma}{\varepsilon_0}
\end{align*}

For problem 2.11, we were tasked with finding the electric field inside and outside a spherical shell with surface charge density $\sigma$. For a distance from the center $z < R$, the electric field is zero and for $z \geq R$ the field is given by 

$$
E = \frac{\sigma}{\varepsilon_0} \frac{R^2}{z^2}
$$

For a small change $\Delta z$ in $R$ we have 

$$
E_{outside} - E_{inside} = \frac{\sigma}{\varepsilon_0}\frac{R^2}{(R + \Delta z)} - 0 = \frac{\sigma}{\varepsilon_0}\frac{R^2}{(R + \Delta z)}
$$

The limit as $\Delta z \rightarrow 0 $ is 

$$
E = \frac{\sigma}{\varepsilon_0}.
$$

\paragraph{(b)} Use Gauss’s law to find the field inside and outside a long hollow cylindrical tube, which carries a uniform surface charge $\sigma$. Check that your result is consistent with Eq. 2.33.

\paragraph{Solution} The field inside the cylindrical tube is zero since any Gaussian surface (in the shape of a cylinder) enclosed by the our cylindrical tube carries no charge. For any distance $z$ where $z$ is greater than or equal to the radius $R$ of the cylindrical tube, the enclosed charge for a length $l$ of the cylindar is 

$$
Q_{enc} = \sigma 2\pi R l.
$$

Applying Gauss's law then gives 

$$
E = \frac{\sigma}{\varepsilon}\frac{R}{z}.
$$

At the boundary ($z = R$) we have 

$$
E_{outside} - E_{inside} = \frac{\sigma}{\varepsilon_0}\frac{R}{R} - 0 = \frac{\sigma}{\varepsilon_0}
$$

\paragraph{(c)} Check that the result of Ex. 2.8 is consistent with boundary conditions 2.34 and 2.36.

\paragraph{Solution} We must show that the potential inside and outside the sphere satisfy the boundary conditions. From the example, we have the following equations. 

\begin{align*}
    V(r) =
\begin{cases}
\frac{Q}{4\pi\varepsilon_0 R}, & 0 \leq r \leq R \quad (\text{Inside the shell}) \\[10pt]
\frac{Q}{4\pi\varepsilon_0 r}, & r > R \quad (\text{Outside the shell})
\end{cases}
\end{align*}

Just outside the sphere we have $r \approx R$. Substituting this into the equation gives the first boundary condition. For the second condition, we must show that the equations satisfy 

$$\frac{\partial V_{above}}{\partial n} - \frac{\partial V_{below}}{\partial n} = -\frac{\sigma}{\varepsilon_0}$$

If we let "below" mean "inside", then we see that $\frac{\partial V_{below}}{\partial n} = 0$ since the potential inside the sphere does change with the normal vector.  Outside the sphere we have

\begin{align*}
    \frac{\partial V_{above}}{\partial n} = \frac{\partial }{\partial n} \frac{Q}{4\pi \varepsilon_0r} = -\frac{Q}{4\pi\varepsilon_0r^2} = -\frac{\sigma}{\varepsilon_0}
\end{align*}

as needed.

\newpage

\subsection*{Problem 2.31}

\begin{figure}[h]
    \centering
    \includegraphics[width=0.5\linewidth]{square-of-charges.png}
\end{figure}

\paragraph{(a)} Three charges are situated at the corners of a square (side a), as shown.  How much work does it take to bring in another charge, $+q$, from far away and place it in the fourth corner?

\paragraph{Solution} The amount of work it would take to bring a positive charge to the remaining corner is 

\begin{align*}
    W = \frac{1}{4\pi\varepsilon_0}\left( \frac{-q^2}{a} + \frac{-q^2}{a} + \frac{q^2}{\sqrt{2}a} \right) = \frac{q^2}{4\pi\varepsilon_0}\frac{1-2\sqrt{2}}{\sqrt{2}a}.
\end{align*}

\paragraph{(b)} How much work does it take to assemble the whole configuration of four charges?

\paragraph{Solution} The amount of work needed to assemble the entire system of four charges is 

\begin{align*}
    W = \frac{1}{4\pi\varepsilon_0} \left( \frac{-q^2}{a} + \frac{q^2}{\sqrt{2}a}  + \frac{-q^2}{a} + \frac{-q^2}{a} + \frac{q^2}{\sqrt{2}a} + \frac{-q^2}{a} \right) = \frac{q^2 (\sqrt{2} - 4)}{4\pi\varepsilon_0 a}
\end{align*}

\newpage
\subsection*{Problem 2.32} Two positive point charges, \(q_A\) and \(q_B\) (with masses \(m_A\) and \(m_B\)), are initially at rest, held together by a massless string of length \(a\). Now the string is cut, and the particles fly off in opposite directions. How fast is each one moving when they are far apart?

\paragraph{Solution} The total work done to assemble this system in the first place is 

\begin{align*}
    W = \frac{1}{4\pi\varepsilon_0} \frac{q_Aq_B}{a}.
\end{align*}

When the string is cut, the charges will fly apart. After a very long time, we expect the total potential energy of the system goes to zero since it decreases as a factor of $1/r$. However, conservation of energy ensures that the potential energy is converted to kinetic energy. Thus, we have 

\begin{align*}
    \frac{1}{4\pi\varepsilon_0} \frac{q_Aq_B}{a} = \frac{1}{2}m_Av_A^2 + \frac{1}{2}m_Bv_B^2.
\end{align*}

We have two unknowns here, so we need another equation. The conservation of momentum helps us.. Since the initial momentum is zero (string hasn't been cut), the final momentum must satisfy 
\begin{align*}
    m_Av_A = -m_Bv_B.
\end{align*}

Solving for $v_A$ gives 

$$
v_A = -\frac{m_Bv_B}{m_A}
$$

Substitution gives 

\begin{align*}
    \frac{1}{4\pi\varepsilon_0} \frac{q_Aq_B}{a} &= \frac{1}{2}m_A\left(\frac{m_Bv_B}{m_A} \right)^2 + \frac{1}{2}m_Bv_B^2 \\
    \frac{1}{4\pi\varepsilon_0} \frac{q_Aq_B}{a} &= \frac{1}{2}v_B^2 \left[\frac{m_B^2}{m_A} + m_B \right]\\
    \implies v_B &= \pm \sqrt{\frac{q_Aq_B}{2\pi\varepsilon_0a\left[\frac{m_B^2}{m_A} + m_B \right]}}
\end{align*}

\pagebreak

If we arbitrarily assign particle $A$ to be moving in the positive direction, then we see that we must take the negative root of $v_B$, so 

$$
v_B = -\sqrt{\frac{q_Aq_B}{2\pi\varepsilon_0a\left[\frac{m_B^2}{m_A} + m_B \right]}}.
$$

Substituting this back into our equation for $v_A$ gives 

$$
v_A = \frac{m_B}{m_A}\sqrt{\frac{q_Aq_B}{2\pi\varepsilon_0a\left[\frac{m_B^2}{m_A} + m_B \right]}}
$$

\newpage

\subsection*{Problem 2.33} Consider an infinite chain of point charges, \( \pm q \) (with alternating signs), arranged along the \( x \)-axis, each separated by a distance \( a \) from its nearest neighbors.

Find the work per particle required to assemble this system.

\textbf{Partial Answer:} The work per particle required to assemble the system is given by:

\[
W = -\frac{\alpha q^2}{4\pi \varepsilon_0 a}
\]

for some dimensionless number \( \alpha \). Your task is to determine \( \alpha \).

\paragraph{Solution} We first draw a diagram of the scenario pictured below. The charge at $x = 0$ has been arbitrarily set to be positive. 

\input{tex_images/infinite-line-of-charge}

The work done to assemble a system of charges is given by 

\begin{align*}
    W = \frac{1}{2} \sum_i^n q_i V(r_i).
\end{align*}

The potential at $x = 0$, denoted $V(x_0)^+$, due to charges at $x > 0$ is given by 

\begin{align*}
    V(x_0)^+ = -\frac{q}{4\pi\varepsilon_0}\frac{1}{a} + \frac{q}{4\pi\varepsilon_0}\frac{1}{2a} - \frac{q}{4\pi\varepsilon_0}\frac{1}{3a} + \dots = \sum_{k = 1}^{\infty} \frac{q}{4\pi\varepsilon_0}\frac{(-1)^k}{ka} = \frac{q}{4\pi\varepsilon_0a}\sum_{k = 1}^{\infty}\frac{(-1)^k}{k}
\end{align*}

Thus, the potential due to all of the charges is just twice as large since the system is completely symmetrical.

\begin{align*}
    V(x_0) = \frac{2q}{4\pi\varepsilon_0a}\sum_{k = 1}^{\infty}\frac{(-1)^k}{k}
\end{align*}

This series has a well known covergence values of $-\ln{2}$. And so at $x = 0$, we have 

$$
V(x_0) = -\frac{2q}{4\pi\varepsilon_0a}\ln{2}.
$$

Because the line of charge is infinitely long, we can take advantage of translational symmetry. This means that the potential at every point containing a charge is the same. This follows because we can choose any point to be $x = 0$, and there will be the same number, namely infinity, of alternating charges on each side of our reference point. 

Substituting this back into the work equation gives

\begin{align*}
    W = -\frac{q}{4\pi\varepsilon_0a}\ln{2}\sum_i^n q_i.
\end{align*}

Since the magnitude of each charge is the same, with only sign differences, we can call $q_i$ constant and pull it out of the sum. 

$$
W = -\frac{q^2}{4\pi\varepsilon_0a}\ln{2}\sum_i^n 1 = n \cdot\left[ -\frac{q^2}{4\pi\varepsilon_0a}\ln{2} \right]
$$

The work done per charge is just $W / n$, so we have 

$$
\text{work done per charge} = -\frac{q^2}{4\pi\varepsilon_0a}\ln{2}.
$$

Comparing with the partial solution given, we see that $\alpha = \ln 2$.

\newpage

\subsection*{Problem 2.34} 
Find the energy stored in a uniformly charged solid sphere of radius $R$ and charge $q$. Do it three different ways.

\paragraph{(a)} Using equation 2.43. You found the potential of this in problem 2.21. 

\paragraph{Solution} The potential of a uniformly charged sphere was found to be 

 \begin{align*}
      U &= \frac{q}{4\pi \varepsilon_0R} - \frac{qr^2}{8\pi\varepsilon_0R^3} + \frac{q}{8\pi\varepsilon_0R} \\
      &= \frac{3q}{8\pi\varepsilon_0R} - \frac{qr^2}{8\pi\varepsilon_0R^3}
 \end{align*}

We need to use the following equation to solve part (a) of the problem. 

$$W = \frac{1}{2}\int \rho V d\tau$$

Because the potential inside the sphere varies with $r$, we must integrate in a way that increments the small sliver of volume $d\tau$ only by the radial variable. A way to do this is to let $d\tau = 4\pi r^2\ dr$. Making these substitutions and integrating from 0 to $R$ gives the result.

\begin{align*}
    W &= \frac{1}{2}\int_0^R \rho \left[ \frac{3q}{8\pi\varepsilon_0R} - \frac{qr^2}{8\pi\varepsilon_0R^3} \right] 4\pi r^2\ dr \\
    \\
    &= \frac{q\rho}{4\varepsilon_0R} \int_0^R \left[3r^2 - \frac{r^4}{R^2} \right]\ dr \\
    \\
    &= \frac{q\rho}{4\varepsilon_0R}\left[ R^3 - \frac{R^5}{5R^2} \right] \\
    \\
    &= \frac{3q^2}{20\pi\varepsilon_0R}
\end{align*}

The last equality is obtained by substituting the equation for $\rho$.


\paragraph{(b)} Use Eq.2.45.  Don't forget to integrate over all space.

\paragraph{Solution} Equation 2.45 gives the energy in a charge distribution as 

\begin{align*}
    E = \frac{\varepsilon}{2}\int E^2 d\tau.
\end{align*}

For this we need to break up the integral into two regions, from 0 to R and from R to infinity.

\begin{align*}
    W = \frac{\varepsilon_0}{2}\int_0^R E^2(r)\ d\tau + \frac{\varepsilon_0}{2}\int_0^{\infty} E^2(r)\ d\tau
\end{align*}

Outside the sphere, $E(r)$ is given by 

\begin{align*}
    E(r) &= \frac{q}{4\pi\varepsilon_0}\frac{1}{r^2} \\
    \implies E^2(r) &= \frac{q^2}{16\pi^2\varepsilon_0^2}\frac{1}{r^4}
\end{align*}

Taking $d\tau = 4\pi r^2\ dr$, the integral from $R$ to infinity is then 

\begin{align*}
    \frac{\varepsilon_0}{2}\int_R^{\infty} E^2(r)\ d\tau &= \frac{\varepsilon_0}{2}\cdot \frac{q^2}{16\pi^2\varepsilon_0^2}\int_R^{\infty} \frac{1}{r^4}4\pi r^2\ dr \\
    \\
    &= \frac{1}{2}\cdot \frac{q^2}{4\pi\varepsilon_0}\int_R^{\infty} \frac{1}{r^2}\ dr = \frac{q^2}{8\pi\varepsilon_0 R}
\end{align*}

Inside the sphere, Gauss's law gives the electric field as 

\begin{align*}
    E(r) &= \frac{q}{4\pi\varepsilon_0}\frac{r}{R^3} \\
    \implies E^2(r) &= \frac{q^2}{16\pi^2 \varepsilon^2_0}\frac{r^2}{R^6} 
\end{align*}

\newpage

Integrating this with the same expression for $d\tau$ gives 

\begin{align*}
    \frac{\varepsilon_0}{2}\int_0^R E^2(r)\ d\tau &= \frac{\varepsilon_0}{2}\cdot \frac{q^2}{16\pi^2 \varepsilon^2_0R^6}\int_0^R 4\pi r^4\ dr \\
    \\
    &= \frac{1}{2}\cdot \frac{q^2}{4\pi \varepsilon_0R^6}\frac{R^5}{5}\ dr = \frac{q^2}{40\pi \varepsilon_0 R}
\end{align*}

Adding these two equations together gives 

\begin{align*}
    W = \frac{q^2}{40\pi \varepsilon_0 R} + \frac{q^2}{8\pi\varepsilon_0 R} = \frac{6q^2}{40\pi\varepsilon_0 R} = \frac{3q^2}{20\pi\varepsilon_0 R}
\end{align*}

\paragraph{(c)} Use Eq. 2.44. Take a spherical volume of radius $a$. What happens as $a \rightarrow \infty$?

\paragraph{Solution} Choose as our integration volume a sphere of radius \( a \) (with \( a>R \)), and later take \( a\to\infty \). Since the charged sphere has radius \( R \), we must break the integration into two regions:
\begin{enumerate}
  \item \textbf{Region I (inside):} \( 0\le r\le R \).
  \item \textbf{Region II (outside):} \( R\le r\le a \).
\end{enumerate}

\textbf{1. Electric Fields and Potential}

For a uniformly charged sphere:
\begin{itemize}
  \item \emph{Inside (\( r\le R \))}: By Gauss’s law,
    \[
    E_{\mathrm{in}}(r)=\frac{q}{4\pi\varepsilon_0}\frac{r}{R^3}.
    \]
  \item \emph{Outside (\( r\ge R \))}: The field is that of a point charge,
    \[
    E_{\mathrm{out}}(r)=\frac{q}{4\pi\varepsilon_0}\frac{1}{r^2}.
    \]
\end{itemize}
Also, for \( r\ge R \) (with \( V(\infty)=0 \)) the potential is
\[
V(r)=\frac{q}{4\pi\varepsilon_0}\frac{1}{r}.
\]

\vspace{2mm}
\textbf{2. Energy Calculation}

\textbf{(a) Energy Inside the Sphere (\( 0\le r\le R \))}

Inside, we have
\[
E_{\mathrm{in}}^2(r)=\left(\frac{q}{4\pi\varepsilon_0}\right)^2 \frac{r^2}{R^6}.
\]
Using the spherical volume element \( d\tau = 4\pi r^2 dr \), the energy is
\begin{align*}
W_{\mathrm{in}} &= \frac{\varepsilon_0}{2}\int_{0}^{R} E_{\mathrm{in}}^2(r)\, d\tau \\
&=\frac{\varepsilon_0}{2}\int_{0}^{R}\left(\frac{q}{4\pi\varepsilon_0}\right)^2 \frac{r^2}{R^6}\,(4\pi r^2 dr) \\
&=\frac{\varepsilon_0}{2}\cdot \frac{q^2}{16\pi^2\varepsilon_0^2R^6}\cdot 4\pi \int_{0}^{R} r^4\, dr \\
&=\frac{q^2}{32\pi\varepsilon_0 R^6}\cdot \frac{R^5}{5} \quad \left(\int_{0}^{R}r^4dr=\frac{R^5}{5}\right)\\[1mm]
&=\frac{q^2}{160\pi\varepsilon_0}\cdot \frac{1}{R} = \frac{q^2}{40\pi\varepsilon_0 R}\,.
\end{align*}

\textbf{(b) Energy Outside the Sphere (\( R\le r\le a \)) – Volume Integral}

Outside, the field is
\[
E_{\mathrm{out}}(r)=\frac{q}{4\pi\varepsilon_0}\frac{1}{r^2} \quad\Longrightarrow\quad E_{\mathrm{out}}^2(r)=\left(\frac{q}{4\pi\varepsilon_0}\right)^2\frac{1}{r^4}.
\]
Then,
\begin{align*}
W_{\mathrm{out,vol}} &= \frac{\varepsilon_0}{2}\int_{R}^{a} E_{\mathrm{out}}^2(r)\, d\tau \\
&=\frac{\varepsilon_0}{2}\int_{R}^{a}\frac{q^2}{16\pi^2\varepsilon_0^2}\frac{1}{r^4}\,(4\pi r^2 dr) \\
&=\frac{q^2}{8\pi\varepsilon_0}\int_{R}^{a}\frac{1}{r^2}\, dr \\
&=\frac{q^2}{8\pi\varepsilon_0}\left[-\frac{1}{r}\right]_{R}^{a} \\
&=\frac{q^2}{8\pi\varepsilon_0}\left(\frac{1}{R}-\frac{1}{a}\right).
\end{align*}

\textbf{(c) Surface Term at \( r=a \)}

At the surface \( r=a \), the potential and field are
\[
V(a)=\frac{q}{4\pi\varepsilon_0}\frac{1}{a} \quad \text{and} \quad E(a)=\frac{q}{4\pi\varepsilon_0}\frac{1}{a^2},
\]
and the area element is \( da = a^2\, d\Omega \) with total area \( 4\pi a^2 \). Therefore,
\begin{align*}
W_{\mathrm{surface}} &= \frac{\varepsilon_0}{2}\oint_{r=a} V(a)\,E(a)\, da \\
&=\frac{\varepsilon_0}{2}\,V(a)\,E(a)\,(4\pi a^2) \\
&=\frac{\varepsilon_0}{2}\left(\frac{q}{4\pi\varepsilon_0 a}\right)\left(\frac{q}{4\pi\varepsilon_0 a^2}\right)(4\pi a^2) \\
&=\frac{q^2}{8\pi\varepsilon_0a}.
\end{align*}

\vspace{2mm}
\textbf{3. Total Energy and the \( a\to\infty \) Limit}

The total energy is the sum of the three contributions:
\begin{align*}
W &= W_{\mathrm{in}} + W_{\mathrm{out,vol}} + W_{\mathrm{surface}} \\
&=\frac{q^2}{40\pi\varepsilon_0 R} + \frac{q^2}{8\pi\varepsilon_0}\left(\frac{1}{R}-\frac{1}{a}\right) + \frac{q^2}{8\pi\varepsilon_0a}\,.
\end{align*}
The \( \frac{1}{a} \) terms cancel:
\[
-\frac{q^2}{8\pi\varepsilon_0a} + \frac{q^2}{8\pi\varepsilon_0a}=0,
\]
so that
\begin{align*}
W &= \frac{q^2}{40\pi\varepsilon_0 R} + \frac{q^2}{8\pi\varepsilon_0 R} \\
&=\frac{q^2}{\pi\varepsilon_0 R}\left(\frac{1}{40}+\frac{1}{8}\right) \\
&=\frac{q^2}{\pi\varepsilon_0 R}\left(\frac{1}{40}+\frac{5}{40}\right) \\
&=\frac{q^2}{\pi\varepsilon_0 R}\cdot\frac{6}{40} \\
&=\frac{3q^2}{20\pi\varepsilon_0 R}.
\end{align*}
The result is then
\[
\boxed{W=\frac{3}{5}\,\frac{q^2}{4\pi\varepsilon_0 R}\,.}
\]

Taking the limit \( a\to\infty \) shows that the extra \( 1/a \) pieces vanish (or cancel between the volume and surface contributions), so the final energy depends only on the physical radius \( R \) of the charged sphere.

\newpage

\paragraph{Problem 2.35}Here is a fourth way of computing the energy of a uniformly charged solid sphere: Assemble it like a snowball, layer by layer, each time bringing in an infinitesimal charge $dq$ from far away and smearing it uniformly over the surface, thereby increasing the radius. How much work $dW$ does it take to build up the radius by an amount $dr$? Integrate this to find the work necessary to create the entire sphere of radius R and total charge q.

\paragraph{Solution} We need to assemble a sphere with radius $R$ and total charge $q$ by building it layer by layer. Assume we have some preassembled sphere of radius $r$. The work done to move a very thin spherical shell onto the sphere is 
\begin{align*}
    dW = V\cdot dq.
\end{align*}
The term $dq$ is given by $dq = \rho 4\pi r^2\ dr$. Using the equation for $\rho = 3q/(4\pi R^3)$ gives $dq = 3q\cdot(r^2/R^3)$.\\

Now we need to find the potential $V$ at the surface of our sphere. We use Gauss's law. We take the potential to be zero at $r = \infty$, so for some distance $z > r$ we have 

\begin{align*}
   &(4\pi z^2) E = \frac{\rho}{\varepsilon}\frac{4\pi r^3}{3} \\ 
   \implies E &= \left(\frac{1}{4\pi z^2}\right) \left(\frac{3q}{4\pi R^3}\right)\left(\frac{4\pi r^3}{3} \right) = \frac{q}{4\pi\varepsilon_0R^3}\frac{r^3}{z^2}
\end{align*}

The potential at $r$ is then the integral from infinity to $r$. 

\begin{align*}
    V(r) = -\int_{\infty}^{r} \frac{q}{4\pi\varepsilon_0R^3}\frac{r^3}{z^2}\ dz = \frac{q}{4\pi\varepsilon_0R^3}r^2
\end{align*}

To find the total work, we integrate again for the expression for $dW$.

\begin{align*}
    W = \int dW &= \int_0^R \left(\frac{q}{4\pi\varepsilon_0R^3}r^2\right)\left(3q\frac{r^2}{R^3}\right)\ dr \\
    \\
    &= \frac{3q^2}{4\pi\varepsilon_0R^6}\int_0^Rr^4\ dr = \frac{3}{5}\frac{q^2}{4\pi\varepsilon_0R}
\end{align*}

This answer is consistent with the other problems.

\newpage

\subsection*{Problem 2.36}  Consider two concentric spherical shells, of radii $a$ and $b$. Suppose the inner one carries a charge $q$, and the outer one a charge $-q$ (both of them uniformly distributed over the surface). 

\paragraph{(a)} Calculate the energy of this configuration using Eq. 2.45.

\paragraph{Solution} We must use the equation 

\begin{align*}
    W = \frac{\varepsilon_0}{2}\int E^2\ d\tau 
\end{align*}

to calculate the energy of the sphere. We need to divide the integral into several regions for this to work. 

\begin{align*}
    W = \frac{\varepsilon_0}{2}\left[ \int_0^a E^2_1\ d\tau + \int_a^b E^2_2\ d\tau + \int_b^{\infty} E^2_3\ d\tau \right]
\end{align*}

By Gauss's law, the first and third integrals are zero because the net enclosed charge in these areas are zero. The problem thus reduces to finding 

\begin{align*}
    W = \frac{\varepsilon_0}{2}\int_b^a E^2 \ d\tau.
\end{align*}

For $a \leq z < b$, we have that the electric field is 

\begin{align*}
    E = \frac{q}{4\pi\varepsilon_0 z^2}
\end{align*}

A small element of volume $d\tau$ can be taken to be $d\tau = 4\pi z^2\ dz$. Substituting and evaluating gives 

$$
W = \frac{\varepsilon_0}{2}\int_a^b \frac{q^2}{16\pi^2\varepsilon_0^2 z^4}4\pi z^2\
 dz = \frac{q^2}{8\pi\varepsilon_0}\int_a^b \frac{1}{z^2}\ dz = \frac{q^2}{8\pi\varepsilon_0}\left(\frac{1}{a} - \frac{1}{b} \right).
$$

\pagebreak

\paragraph{(b)} Using Eqn. 2.47 and the results of Ex. 2.9.

\paragraph{Solution} Equation 2.47 states that 

\begin{align*}
    W = W_1 + W_2 + \varepsilon_0\int \vec{E}_1\cdot \vec{E}_2\ d\tau
\end{align*}

and example 2.9 gives the energy of a sphere as 

\begin{align*}
    W = \frac{q^2}{8\pi\varepsilon_0R}
\end{align*}

The energy needed to assemble the inner sphere and outer sphere are then 

\begin{align*}
    W_{inner} = \frac{q^2}{8\pi\varepsilon_0a} \quad W_{outer} = \frac{q^2}{8\pi\varepsilon_0b}
\end{align*}

To calculate the interaction terms, we note that for $a \leq z < b$, the term $\vec{E}_1 \cdot \vec{E}_2 = 0$ since in this region $\vec{E}_2 = 0$, so we only need to focus on the region beyond $b$. Here we note that the vectors always point in the same direction with the same magnitude by Gauss's law (each sphere can be be considered a point charge at the origin).

\begin{align*}
    \vec{E}_1 \cdot \vec{E}_2 &= \left\langle \frac{q}{4\pi\varepsilon_0}\frac{1}{r^2}\bvec{r},  \frac{-q}{4\pi\varepsilon_0}\frac{1}{r^2}\bvec{r}\right\rangle \\
    \\
    &= \frac{-q^2}{16\pi^2 \varepsilon_0^2}\frac{1}{r^4}\langle \bvec{r}, \bvec{r}\rangle = \frac{-q^2}{16\pi^2 \varepsilon_0^2}\frac{1}{r^4}
\end{align*}

Integrating gives 

\begin{align*}
   W_{int} = \varepsilon_0 \int_b^{\infty} d\tau &= \varepsilon_0 \int_b^{\infty} \frac{-q^2}{16\pi^2 \varepsilon_0^2}\frac{1}{r^4}4\pi r^2\ dr  = \frac{-q^2}{4\pi \varepsilon_0} \int_b^{\infty} \frac{1}{r^2}\ dr = -\frac{-q^2}{4\pi \varepsilon_0}\frac{1}{b}
\end{align*}

Summing all of these terms ($W_{total} = W_{inner} + W_{outer} + W_{int}$) gives the same result as part (a).

\newpage

\subsection*{Problem 2.37}
Find the interaction energy ( $\int \vec{E}_1 \cdot \vec{E}_2\ d\tau$ in Eq. 2.47) for two point charges, $q_1$ and $q_2$, a distance a apart. 

\paragraph{Solution} We set up a diagram of the situation below with charge $q_1$ at the origin and charge $q_2$ a distance $a$ away on the $z$ axis. 

\begin{center}
    \input{tex_images/interaction-energy}
\end{center}

We need to derive expressions for $\vec{E}_1$ and $vec{E}_2$. For the first one we have the electric field due to $q_1$ is 

\begin{align*}
    \vec{E}_1 = \frac{q_1}{4\pi\varepsilon_0}\frac{1}{|\vec{r}_1|^2}\hat{r}_1.
\end{align*}

For $\vec{E}_2$ we have 

\begin{align*}
    \vec{E}_2 = \frac{q_2}{4\pi\varepsilon_0}\frac{1}{|\vec{r}_2|^2}\hat{r}_2.
\end{align*}

\pagebreak

The integral we need to evaluate is then 

\begin{align*}
    \int \vec{E}_1 \cdot \vec{E}_2\ d\tau &= \int \left\langle \frac{q_1}{4\pi\varepsilon_0}\frac{1}{|\vec{r}_1|^2}\hat{r}_1, \frac{q_2}{4\pi\varepsilon_0}\frac{1}{|\vec{r}_2|^2}\hat{r}_2 \right\rangle d\tau = \frac{q_1q_2}{16\pi^2\varepsilon^2_0} \int \frac{1}{|\vec{r}_1|^2}\frac{1}{|\vec{r}_2|^2} \left\langle \hat{r}_1, \hat{r}_2 \right\rangle d\tau
\end{align*}

Using the fact that $\hat{r}_1 = \vec{r}_1/|\vec{r}_1|$ and $\hat{r}_2 = \vec{r}_2/|\vec{r}_2|$ we have the following. 

\begin{align*}
    \int \vec{E}_1 \cdot \vec{E}_2\ d\tau = \frac{q_1q_2}{16\pi^2\varepsilon^2_0} \int \frac{1}{|\vec{r}_1|^3}\frac{1}{|\vec{r}_2|^3} \left\langle \vec{r}_1, \vec{r}_2 \right\rangle d\tau
\end{align*}

We now need to determine the form of $\vec{r}_1$ and $\vec{r}_2$. In spherical coordinates, $\rvec{1}$ can be written as 

\begin{align*}
    \rvec{1} = |\rvec{1}|\sin\theta\cos\phi\cdot\xvec + |\rvec{1}|\sin\theta\sin\phi\cdot\yvec + |\rvec{1}|\cos\theta\cdot\zvec.
\end{align*}

The vector $\rvec{2}$ can be formulated in terms of $\rvec{1}$ by noticing that it is just the same vector with a difference in the $z$ coordinate. So, $\rvec{2}$ is 

\begin{align*}
    \rvec{2} = \rvec{1} - a\cdot \zvec = |\rvec{1}|\sin\theta\cos\phi\cdot\xvec + |\rvec{1}|\sin\theta\sin\phi\cdot\yvec + (|\rvec{1}|\cos\theta - a)\cdot\zvec
\end{align*}

Taking the dot product of these two vectors gives 

\begin{align*}
\vec{r}_1 \cdot \vec{r}_2 
  &= \Bigl(|\vec{r}_1|\sin\theta\cos\phi\,\hat{x} 
      + |\vec{r}_1|\sin\theta\sin\phi\,\hat{y} 
      + |\vec{r}_1|\cos\theta\,\hat{z}\Bigr) \cdot \\
  &\quad\Bigl(|\vec{r}_1|\sin\theta\cos\phi\,\hat{x} 
      + |\vec{r}_1|\sin\theta\sin\phi\,\hat{y} 
      + \bigl(|\vec{r}_1|\cos\theta - a\bigr)\hat{z}\Bigr)\\
  &= \Bigl(|\vec{r}_1|\sin\theta\cos\phi\Bigr)
      \Bigl(|\vec{r}_1|\sin\theta\cos\phi\Bigr) + \Bigl(|\vec{r}_1|\sin\theta\sin\phi\Bigr)
      \Bigl(|\vec{r}_1|\sin\theta\sin\phi\Bigr) + \Bigl(|\vec{r}_1|\cos\theta\Bigr)
      \Bigl(|\vec{r}_1|\cos\theta - a\Bigr) \\
  &= |\vec{r}_1|^2 \sin^2\theta\cos^2\phi 
      + |\vec{r}_1|^2 \sin^2\theta\sin^2\phi + |\vec{r}_1|^2\cos^2\theta - a\,|\vec{r}_1|\cos\theta.
\end{align*}

Notice that
\[
\sin^2\theta\cos^2\phi + \sin^2\theta\sin^2\phi 
  = \sin^2\theta \left(\cos^2\phi + \sin^2\phi\right) 
  = \sin^2\theta.
\]

Substituting this result back in, we obtain:
\begin{align*}
\vec{r}_1 \cdot \vec{r}_2 
  &= |\vec{r}_1|^2 \Bigl(\sin^2\theta + \cos^2\theta\Bigr) 
      - a\,|\vec{r}_1|\cos\theta \\
  &= |\vec{r}_1|^2 (1) - a\,|\vec{r}_1|\cos\theta = |\vec{r}_1|^2 - a\,|\vec{r}_1|\cos\theta.
\end{align*}

Substituting this back into the integral gives 

\begin{align*}
    \int \vec{E}_1 \cdot \vec{E}_2\ d\tau = \frac{q_1q_2}{16\pi^2\varepsilon^2_0} \int \frac{1}{|\vec{r}_1|^3}\frac{1}{|\vec{r}_2|^3} (|\vec{r}_1|^2 - a\,|\vec{r}_1|\cos\theta) d\tau
\end{align*}

We now need to write $|\rvec{2}|$ in terms of $|\rvec{1}|$. Using the diagram and the law of cosines, we have the following relationship.

\begin{align*}
    |\rvec{2}| = a^2 + |\rvec{1}|^2 - 2a|\rvec{1}|\cos\theta.
\end{align*}

We also note that $d\tau = |\rvec{1}|^2\sin\theta\ dr\ d\theta \ d\phi$. Since everything is now in terms of $\rvec{1}$, we can set $r = |\rvec{1}|$ and substitute the expressions obtained to write the integral we need to evaluate. 

\begin{align*}
    \int \vec{E}_1 \cdot \vec{E}_2\ d\tau &= \frac{q_1q_2}{16\pi^2\varepsilon^2_0} \int \frac{r^2 - a\,r\cos\theta}{r^3(a^2 + r^2 - 2ar\cos\theta)^{3/2}} r^2\sin\theta\ dr\ d\theta d\phi\\
\\
    &= \frac{q_1q_2}{16\pi^2\varepsilon^2_0} \int_0^{2\pi}\int_0^{\pi}\int_{0}^{\infty} \frac{r - a\cos\theta}{(a^2 + r^2 - 2ar\cos\theta)^{3/2}}\sin\theta\ dr\ d\theta d\phi \quad (\text{Cancel out } r)
\end{align*}

The inner integral can be performed using $u$ substitution be setting $u = a^2 + r^2 - 2ar\cos\theta$. Integrating this gives 

\begin{align*}
    \frac{q_1q_2}{16\pi^2\varepsilon^2_0} \int_0^{2\pi}\int_0^{\pi} -\frac{1}{a} \sin\theta \ d\theta d\phi &= \frac{q_1q_2}{16\pi^2\varepsilon^2_0} \int_0^{2\pi}-\frac{1}{a}[\cos\theta]\big|^{\pi}_0 \\
    &=\frac{q_1q_2}{16\pi^2\varepsilon^2_0} \int_0^{2\pi}\frac{2}{a}\ d\phi \\
    &= \frac{q_1q_2}{16\pi^2\varepsilon^2_0} \frac{4\pi}{a}\\
    &= \frac{q_1q_2}{4\pi\varepsilon^2_0} \frac{1}{a}\\
\end{align*}


Recalling that the interaction energy is multiplied by $\varepsilon_0$ we have the final interaction energy below. 

\begin{align*}
    \boxed{\frac{q_1q_2}{4\pi\varepsilon_0} \frac{1}{a}}
\end{align*}

\newpage

\subsection*{Problem 2.38} A metal sphere of radius $R$, carrying charge $q$, is surrounded by a thick concentric metal shell (inner radius a, outer radius b). The shell carries no net charge.

\paragraph{(a)} Find the surface charge density $\sigma$ at $R$, at $a$, and at $b$. 

\paragraph{Solution} Because the sphere is a metal, it holds all charge on the surface. Thus, the surface charge density at $R$ is $\sigma_R = q/(4\pi R^2)$.\\

The charge on the surface of the sphere will induce a total charge $-q$ on the inner radius of the spherical shell. Thus, the surface charge density here is $\sigma_a = -q/(4\pi a^2)$. \\

The induced charge at $a$ will cause a positively induced charge at $b$, so that the surface charge density here is $\sigma_b = q/(4\pi b^2)$.

\paragraph{(b)} Find the potential at the center, using infinity as the reference point.

\paragraph{Solution} We have four regions we need to consider. 
\begin{align*}
    \text{Region IV}: r = b \quad \text{Region III}: a \leq r < b \quad \text{Region II}: R \leq r < a \quad \text{Region I}: r < R
\end{align*}

For Region IV, the potential is given by at $b$ is given by

\begin{align*}
    V(b) = -\int_{\infty}^b \frac{q}{4\pi \varepsilon}\frac{1}{r^2}\ dr = \frac{q}{4\pi\varepsilon_0}\frac{1}{b}
\end{align*}

By Gauss's law, for $a \geq r < b$, we have $E = 0$ since the total enclosed charge is zero, thus the potential is actually the same as at $b$ since we have 

\begin{align*}
    V(a) = -\int_{\infty}^b \frac{q}{4\pi\varepsilon_0}\frac{1}{r}\ dr - \int_b^a 0 \ dr = \frac{q}{4\pi \varepsilon_0}\frac{1}{b}
\end{align*}

The potential at the surface of the solid sphere is given by 

\begin{align*}
    V(R) &= -\int_{\infty}^b \frac{q}{4\pi\varepsilon_0}\frac{1}{r}\ dr - \int_b^a 0 \ dr - \int_a^R \frac{q}{4\pi \varepsilon_0}\frac{1}{r^2}\ dr = \frac{q}{4\pi \varepsilon_0}\frac{1}{b} + \frac{q}{4\pi \varepsilon_0}\left[ \frac{1}{R} - \frac{1}{a}\right].
\end{align*}

Then, the potential at the center is the potential at the surface $R$. This is because the net enclosed charge within the sphere is zero, and so there is nothing to contribute to the potential in the region.

\paragraph{(c)} Now the outer surface is touched to a grounding wire, which drains off charge and lowers its potential to zero (same as at infinity). How do your answers to (a) and (b) change?

\paragraph{Solution} In this scenario, our potential calculation changes slightly. Outside of $a$, or $b$ for that matter, the electric field is zero and so it takes no energy to bring a positive test charge from infinity to either $a$ or $b$. Thus, the surface charge density at $b$ is zero and the potential at the center of the sphere is 

\begin{align*}
    V(0) = \frac{q}{4\pi \varepsilon_0}\left[ \frac{1}{R} - \frac{1}{a}\right]
\end{align*}

\newpage

\subsection*{Problem 2.39}
Two spherical cavities, of radii $a$ and $b$, are hollowed out from the interior of a (neutral) conducting sphere of radius $R$. At the center of each cavity a point charge is placed - call these charges $q_a$ and $q_b$.

\begin{figure}[h]
    \centering
    \includegraphics[width=0.35\linewidth]{images/cavities-in-spheres.png}
\end{figure}

\begin{enumerate}[label=(\alph*)]
    \item Find the surface charge densities $\sigma_R, \sigma_a, \sigma_b$.
    \item What is the field outside the conductor?
    \item What is the field within each cavity?
    \item What is the force on $q_a$ and $q_b$?
\end{enumerate}

\paragraph{Solution} To find the surface charge densities, we recognize that the charge in each cavity induces a charge of equal magnitude but opposite sign on the inner edge. Thus, the surface charge densities are 

\begin{align*}
    \sigma_a = \frac{-q_a}{4\pi a^2} \quad \sigma_b = \frac{-q_b}{4\pi b^2} \quad \sigma_R = \frac{q_a + q_b}{4\pi R^2}
\end{align*}

The field outside of the conductor is given by Gauss's law. 

\begin{align*}
    E(r) = \frac{q_a + q_b}{4\pi\varepsilon_0 r^2}
\end{align*}


The field within each cavity is also given by Gauss's law.

\begin{align*}
    E_a &= \frac{q_a}{4\pi \varepsilon_0 z^2} \quad \text{for } z < a \\
    \\
    E_b &= \frac{q_b}{4\pi \varepsilon_0 z^2} \quad \text{for } z < b
\end{align*}

Finally, the force on each charge is zero by symmetry since the charges are placed in the center of each cavity.

\newpage

\subsection*{Problem 2.40} 

\paragraph{(a)} A point charge $q$ is inside a cavity in an uncharged conductor.  Is the force on $q$ necessarily zero?

\paragraph{Solution} This is not necessarily true. It is only true for problems that exhibit symmetry. If a point charge is in a non-symmetric cavity, it will generally experience a force.

\paragraph{(b)} Is the force between a point charge and a nearby uncharged conductor always attractive?

\paragraph{Solution} This is true by definition of induced charge. Any negative charge will repel electrons in the nearby conductor, leaving a closer net positive charge. Any positive charge will attract electrons from the nearby conductor.

\newpage

\subsection*{Problem 2.41}  Two large metal plates (each of area $A$) are held a small distance $d$ apart. Suppose we put a charge $Q$ on each plate; what is the electrostatic pressure on the plates?

\paragraph{Solution} The situation can be described in the diagram below. 
\medskip
\input{tex_images/pressure-on-plates}

The field in between the plates cancels since each plate carries the same charge. On either side of the plate, the field is twice as strong and given by $E = 2\frac{\sigma}{2\varepsilon_0} = \frac{\sigma}{\varepsilon_0}$. The pressure is then $P = (\varepsilon_0/2)(\sigma^2/\varepsilon_0^2) = \sigma^2/2\varepsilon_0$.

\newpage

\subsection*{Problem 2.42}
A metal sphere of radius $R$ carries a total charge $Q$. What is the force of repulsion between the “northern” hemisphere and the “southern” hemisphere?

\paragraph{Solution} We first note that since the sphere is made of metal, all charge sits on the outside and the electric field on the interior is zero. The field just outside the sphere is 

\begin{align*}
    E = \frac{Q}{4\pi\varepsilon_0}\frac{1}{R^2} = \frac{\sigma}{\varepsilon_0}
\end{align*}

The force per unit area due to other fields (the electric fields produced by other patches of the sphere) is 

\begin{align*}
    f = \frac{\sigma^2}{2\varepsilon_0}\cdot \bvec{n}
\end{align*}

The magnitude of the repulsion force (i.e. the force pointing away from the other hemisphere) is given by integrating the vertical component of the normal vector over the hemisphere. 

\begin{align*}
    F = \int_0^{\pi/2} \int_0^{2\pi} \frac{\sigma^2}{2\epsilon_0}\cos{\theta}\sin{\theta}R^2d\phi d\theta
\end{align*}

Using $\cos\theta \sin\theta = \frac{1}{2}\sin2\theta$ and integrating with respect to $\phi$ first we have 

\begin{align*}
    F = \frac{\sigma^2\pi}{2\varepsilon_0}\int_0^{\pi/2}\sin2\theta\ d\theta = \frac{\sigma^2\pi}{2\varepsilon_0} = \frac{Q^2}{32\pi\varepsilon_0 R^2}
\end{align*}

\newpage

\subsection*{Problem 2.43}
Find the capacitance per unit length of the coaxial metal tubes of radii $a$ and $b$ where $a < b$.

\begin{figure}[h]
    \centering
    \includegraphics[width=0.5\linewidth]{coaxial-capacitance.png}
\end{figure}

\paragraph{Solution} We imagine that we put a $+Q$ charge on the inner cylinder and a $-Q$ charge on the outer cylinder. We need to find the potential difference between the surface of the two cylinders. Gauss's law gives the field between the two cylinders as 

\begin{align*}
    E = \frac{Q}{2\pi z l \varepsilon_0} \quad \text{for } a\leq z < b
\end{align*}

If $+Q$ is on the inner cylinder, then it is at a higher potential. Thus, the work the field does to move a charge from $a$ to $b$ is

\begin{align*}
    V(a) - V(b) = \int_{a}^{b} E\ dz = \frac{Q}{2\pi l\varepsilon_0}\int_a^b \frac{1}{z}\ dz = \frac{Q}{2\pi l\varepsilon_0}\left[ \ln{b} - \ln{a} \right] = \frac{Q}{2\pi l\varepsilon_0}\ln\frac{b}{a}
\end{align*}

The capacitance is then 

\begin{align*}
    C = \frac{Q}{\Delta V} = \frac{2\pi l \varepsilon_0}{\ln{b/a}}
\end{align*}

The capacitance per unit length is 

\begin{align*}
    \frac{C}{l} = \frac{2\pi\varepsilon_0}{\ln{b/a}}
\end{align*}

\newpage

\subsection*{Problem 2.44}
 Suppose the plates of a parallel-plate capacitor move closer together
 by an infinitesimal distance, as a result of their mutual attraction.

\paragraph{(a)} Use Eq. 2.52 to express the work done by electrostatic forces, in terms of the field $E$, and the area of the plates, $A$.

\paragraph{Solution} Equation 2.52 gives the pressure due to electrostatic forces as 

\begin{align*}
    P = \frac{\varepsilon_0}{2}E^2
\end{align*}

Thus, if the capacitors move closer together by a distance $\epsilon$, then the work done by the field is 

\begin{align*}
    F = P\cdot A = A \int_0^{\epsilon} \frac{\varepsilon_0}{2}E^2 = \frac{\varepsilon_0}{2}AE^2\epsilon.
\end{align*}

\paragraph{(b)} Use Eq. 2.46 to express the energy lost by the field in this process.

\paragraph{Solution} The energy per unit volume of a charge configuration is $(\varepsilon_0/2)E^2$. For plates of area $A$, the change in volume moving a slight distance $\epsilon$ is given by

\begin{align*}
    \Delta V = -A\epsilon
\end{align*}

The total change in potential energy for the configuration is 

\begin{align*}
    \text{Energy per Unit Volume} \times \text{Change in Volume} = -\frac{\varepsilon_0}{2}E^2A\epsilon
\end{align*}

\newpage

\subsection*{Problem 2.45} Find the electric field at a height $z$ above the center of a square sheet (side $a$) carrying a uniform surface charge $\sigma$. Check your result for the limiting cases $\lim {a\rightarrow \infty}$ and $z >> a$.

\paragraph{Solution} We set the center of the square sheet to be at the origin so that $x$ and $y$ both vary from $-a/2$ to $a/2$. The distance $r$ from any point on the plane to the point of interest is given by 

\begin{align*}
    r = \sqrt{x^2 + y^2 + z^2}
\end{align*}

The contribution of a small element $dA = dxdy$ on the sheet located as $(x, y)$ to the vertical component of the electric field is given by 

\begin{align*}
    dE_z = dE \sin\theta = \frac{1}{4\pi\varepsilon_0}\frac{\sigma dx\cdot dy}{x^2 + y^2 + z^2} \frac{z}{\sqrt{x^2 + y^2 + z^2}} = \frac{\sigma}{4\pi\varepsilon_0}\frac{z}{(x^2 + y^2 + z^2)^{3/2}}dx \ dy
\end{align*}

Integrating over all $x$ and $y$ gives 

\begin{align*}
    E_z = \frac{z\sigma}{4\pi\varepsilon_0}\int_{-a/2}^{a/2} \int_{-a/2}^{a/2} \frac{1}{(x^2 + y^2 + z^2)^{3/2}}dx \ dy
\end{align*}

Evaluating the first integral using trigonometric substitution gives 

\begin{align*}
    E_z = \frac{z\sigma}{4\pi\varepsilon_0}\int_{-a/2}^{a/2} \frac{2a}{\left(z^{2} + y^{2}\right) \sqrt{4z^{2} + 4y^{2} + a^{2}}}\ dy
\end{align*}

Integrating again gives 

\begin{align*}
    E_z = \frac{\sigma}{2\pi\varepsilon_0} \left[\arctan\left(\frac{ay}{z \sqrt{4 \left(y^{2} + z^{2}\right) + a^{2}}}\right)\right]^{a/2}_{-a/2}
\end{align*}

We note that $\arctan(-x) = -\arctan(x)$ to that 

\begin{align*}
    E_z = \frac{\sigma}{\pi \varepsilon_0}\arctan\left(\frac{a^2}{2z \sqrt{4 z^{2} + 2a^{2}}}\right).
\end{align*}

We check the validity of this solution by allowing $a \rightarrow \infty$. In this case, the solution approaches

\begin{align*}
    \lim_{a \rightarrow \infty} E_z = \lim_{a \rightarrow \infty} \frac{\sigma}{\pi \varepsilon_0}\arctan\left(\frac{a^2}{2z \sqrt{4 z^{2} + 2a^{2}}}\right) = \frac{\sigma}{\pi\varepsilon_0}\frac{\pi}{2} = \frac{\sigma}{2\varepsilon_0}
\end{align*}

which is the result we get from using Gauss's law. If $z >> a$, then the $\arctan$ term approaches zero and so does the strength of the field. This aligns with intuition. 


\newpage

\subsection*{Problem 2.46}
If the electric field in some region is given (in spherical coordinates) by the expression

\begin{align*}
    \mathbf{E}(\mathbf{r}) = \frac{k}{r}\left[ 3\cdot \bvec{r} + 2\sin\theta\cos\theta\sin\phi\cdot \bvec{\theta} + \sin\theta\cos\phi \cdot \bvec{\phi} \right]
\end{align*}

for some constant $k$, what is the charge density?

\newpage

\subsection*{Problem 2.46}
If the electric field in some region is given (in spherical coordinates) by the expression 

\begin{align*}
    E(r) =\frac{k}{r}[3\bvec{r} + 2\sin\theta\cos\theta\sin\phi \bvec{\theta} + \sin\theta\cos\phi \bvec{\phi}]
\end{align*}

\paragraph{Solution} The divergence of any electrostatic field $E$ is equal to the charge density divided by the vacuum permittivity.  In spherical coordinates, the divergence is given by 

\[
\nabla \cdot \mathbf{E} = \frac{1}{r^2}\frac{\partial}{\partial r}\left(r^2 E_r\right)
+\frac{1}{r\sin\theta}\frac{\partial}{\partial \theta}\left(\sin\theta\, E_\theta\right)
+\frac{1}{r\sin\theta}\frac{\partial E_\phi}{\partial \phi} = \frac{\rho}{\varepsilon_0}
\]


We compute the terms below\\


\begin{align*}
    \frac{1}{r^2}\frac{\partial}{\partial r}\left(r^2 E_r\right) = \frac{1}{r^2}\frac{\partial}{\partial r}(3r^2) = \frac{1}{r^2}(6r) = \frac{6}{r}
\end{align*}

\begin{align*}
    \frac{1}{r\sin\theta}\frac{\partial}{\partial \theta}\left(\sin\theta E_\theta\right) &= \frac{1}{r\sin\theta}\frac{\partial }{\partial \theta}(2\sin^2\theta\cos\theta\sin\phi) \\
    \\
    &= \frac{2\sin\phi}{r\sin\theta}\frac{\partial}{\partial \theta}
    (\sin^2\theta\cos\theta)\\
    \\
    &= \frac{2\sin\phi}{r\sin\theta}\frac{\partial}{\partial \theta}(\cos\theta - \cos^3\theta)\\
    \\
    &= \frac{2\sin\phi}{r\sin\theta}\left[-\sin\theta + 3\cos^2\theta\sin\theta\right]\\
    \\
    &= \frac{6\cos^2\theta\sin\phi - \sin\phi}{r}
\end{align*}

\begin{align*}
    \frac{1}{r\sin\theta}\frac{\partial E_\phi}{\partial \phi} &= \frac{1}{r\sin\theta} \frac{\partial }{\partial \phi}(\sin\theta\cos\phi)\\
    \\
    &= -\frac{1}{r\sin\theta}\sin\theta\sin\phi\\
    \\
    &= -\frac{\sin\phi}{r}
\end{align*}

combining these we have 

\begin{align*}
    \nabla \cdot E &= \frac{k}{r}\left[ \frac{6}{r} +  \frac{6\cos^2\theta\sin\phi - \sin\phi}{r} -\frac{\sin\phi}{r}\right]\\
    \\
    &=\frac{k}{r^2}\left[6 + 6\cos^2\theta\sin\phi - 2\sin\phi\right]
\end{align*}

The charge density is then 

\begin{align*}
    \rho = \frac{k\varepsilon_0}{r^2}\left[6 + 6\cos^2\theta\sin\phi - 2\sin\phi\right]
\end{align*}

\newpage

\subsection*{Problem 2.47}
 Find the net force that the southern hemisphere of a uniformly
 charged solid sphere exerts on the northern hemisphere. Express your answer in
 terms of the radius $R$ and the total charge $Q$.

\paragraph{Solution} We need to find an expression for the electric field inside the sphere. By Gauss's law, we have 

\begin{align*}
    \oint_S E\cdot d\vec{a} = \frac{Q_{enc}}{\varepsilon_0}
\end{align*}

The field exhibits radial symmetry so that for a given $r \leq R$ the equation reduces to 

\begin{align*}
    E(4\pi r^2) &= \frac{\rho}{\varepsilon_0}\left(\frac{4}{3}\pi r^3\right) \\
    E(4\pi r^2) &= \frac{1}{\varepsilon_0}\left(\frac{3Q}{4\pi R^3}\right)\left(\frac{4}{3}\pi r^3\right) \\
    \implies E &= \frac{Q}{4\pi\varepsilon_0}\frac{r}{R^3}
\end{align*}

For a small volume element $d\tau$ a distance $r$ away from the center experiences a force $dF = dq\cdot E$. The amount of charge $dq$ in volume $d\tau$ is $da = \rho\cdot d\tau$. The contribution to the total force that the northern hemisphere experiences in the $z$ direction is 

\begin{align*}
    dF_z &= dF\cos{\theta} = E\rho\ d\tau \cos{\theta} \\
    \\
    &= \left(\frac{Q}{4\pi\varepsilon_0}\frac{r}{R^3}\right)\left(\frac{3Q}{4\pi R^3}\right)\cos{\theta}\ d\tau\\
    \\
    &= \frac{3Q^2}{16\pi^2\varepsilon_0}\frac{r}{R^6}\cos{\theta}d\tau
\end{align*}

\pagebreak

Integrating over the volume gives 

\begin{align*}
    F_z = \frac{3Q^2}{16\pi^2\varepsilon_0}\frac{1}{R^6}\int_V r\cos{\theta}d\tau &= \frac{3Q^2}{16\pi^2\varepsilon_0}\frac{1}{R^6}\int_0^{\
    pi/2}\int_{0}^{2\pi}\int_0^R r\cos{\theta}(r^2\sin\theta\ dr\ d\phi\ d\theta) \\
    \\
    &= \frac{3Q^2}{16\pi^2\varepsilon_0}\frac{1}{R^6} \int_0^{\
    pi/2}\int_{0}^{2\pi}\int_0^R r^3\cos{\theta}\sin{\theta}\ dr\ d\phi \ d\theta\\
    \\
    &=\frac{3Q^2}{16\pi^2\varepsilon_0}\frac{1}{R^6}\frac{R^4}{4} 2\pi \int_0^{\pi/2}\frac{1}{2}\sin{2\theta} d\theta \\
    \\
    &= \frac{3Q^2}{16\pi^2\varepsilon_0}\frac{1}{R^6}\frac{R^4}{4} 2\pi \frac{1}{2}\left[-\frac{1}{2}\cos{2\pi}\right]^{\pi/2}_0\\
    \\
    &= \frac{1}{4\pi\varepsilon_0R^2}\frac{3Q^2}{16}
\end{align*}

\newpage


\subsection*{Problem 2.48}
 An inverted hemispherical bowl of radius $R$ carries a uniform surface
 charge density $\sigma$. Find the potential difference between the “north pole” and the center.

 \paragraph{Solution} The situation can be described by the following diagram. 

 \input{tex_images/inverted-spherical-bowl}

To find the potential at $N$, we compute the small contribution of a patch $d\vec{a}$ to the potential at that point. 

\begin{align*}
    dV(N) = \frac{1}{4\pi\varepsilon_0}\frac{\sigma d\vec{a}}{|\vec{r}|} = \frac{1}{4\pi\varepsilon_0}\frac{\sigma d\vec{a}}{\sqrt{2R^2 - 2R^2\cos{\theta}}}
\end{align*}

The last equality follows from the law of cosines. Integrating in spherical coordinates we have 

\begin{align*}
    V(N) &= \frac{\sigma}{4\pi\varepsilon_0}\int_0^{\pi/2}\int_0^{2\pi} \frac{R^2\sin\theta}{\sqrt{2R^2 - 2R^2\cos{\theta}}}\ d\tau\ d\theta \\
    \\
    &= \frac{\sigma R}{4\pi\varepsilon_0} \int_0^{\pi/2}\int_0^{2\pi}\frac{\sin\theta}{\sqrt{2(1 - \cos\theta)}}\ d\tau \ d\theta\\
    \\
    &= \frac{\sigma R}{2\varepsilon_0}\int_0^{\pi / 2} \frac{\sin\theta}{\sqrt{2(1 - \cos\theta)}}\ d \theta\\
    \\
    &\text{let } u = 1 - \cos\theta \quad \implies \quad du = \sin\theta\ d\theta \\
    \\
    &= \frac{\sigma R}{2\sqrt{2}\varepsilon_0}\int_0^{1} \frac{1}{\sqrt{u}}\ du = \frac{\sqrt{2}R}{2}\frac{\sigma}{\varepsilon_0}
\end{align*}

The potential as the center of the base is easy to calculate since it is equidistant from each point on the surface. A small path $da$ contributes 

\begin{align*}
    dV(O) &= \frac{1}{4\pi\varepsilon_0}\frac{\sigma \cdot da}{R}\\
    \\
    \implies V(O) &= \frac{\sigma}{4\pi\varepsilon_0R}\int_0^{\pi/2}\int_0^{2\pi}R^2\sin\theta d\tau d\theta\\ \\
    & = \frac{\sigma R}{4\pi \varepsilon_0}2\pi = \frac{\sigma R}{2\varepsilon_0}
\end{align*}

The potential difference between $O$ and $N$ is 

\begin{align*}
    V(N) - V(O) = \frac{\sigma R}{2\varepsilon_0}(\sqrt{2} - 1)
\end{align*}

\newpage

\subsection*{Problem 2.49} A sphere of radius $R$ carries a charge density $\rho(r) = kr$ (where k is a constant). Find the energy of the configuration. Check your answer by calculating it in at least two different ways.

\paragraph{Solution} The energy of a system of charges is given by 

\begin{align*}
    W = \frac{\varepsilon_0}{2}\int E^2
\end{align*}

where the integral is taken over all space. We need to divide the problem into two regions. One inside the sphere and one outside the sphere. To find $E$ inside the sphere.

\begin{align*}
    W = \frac{\varepsilon_0}{2}\int_0^R E^2 + \frac{\varepsilon_0}{2}\int_R^{\infty} E^2
\end{align*}

For the first integral, we use Gauss's law. At a radius $r \leq R$ the enclosed charge is 

\begin{align*}
    Q_{enc} = \int_0^r \rho 4\pi z^2\ dz = \int_0^r 4\pi k z^3\ dz = k\pi r^4.
\end{align*}

The electric field at a point $r$ away from the center is 

\begin{align*}
    E = \frac{1}{\varepsilon_0}\frac{1}{4\pi r^2}k\pi r^4 = \frac{k}{4\varepsilon_0}r^2
\end{align*}

The first integral thus becomes 

\begin{align*}
    \frac{\varepsilon_0}{2} \int_0^R \left( \frac{k}{4\varepsilon_0}r^2 \right)^2 4\pi r^2\ dr &= \frac{4\pi k^2}{32\varepsilon_0}\int_0^R r^6\ dr = \frac{4\pi k^2}{32\varepsilon_0}\frac{R^7}{7}
\end{align*}

For the second integral, the field is the same as if all charges were concentrated at the center. For $r > R$, Gauss's law gives

\begin{align*}
    E = \frac{1}{4\pi r^2}\frac{k\pi R^4}{\varepsilon_0} = \frac{k}{\varepsilon_0}\frac{R^4}{4r^2}
\end{align*}


The second integral is then 

\begin{align*}
    \frac{\varepsilon_0}{2}\int_R^{\infty} \left(  \frac{k}{\varepsilon_0}\frac{R^4}{4r^2} \right)^24\pi r^2\ dr = \frac{4\pi k^2R^8}{32\varepsilon_0}\int_R^{\infty}\frac{1}{r^2}\ dr = \frac{4\pi k^2R^7}{32\varepsilon_0}
\end{align*}

Adding these together gives 

\begin{align*}
    W = \frac{4\pi k^2}{32\varepsilon_0}\frac{R^7}{7} + \frac{4\pi k^2R^7}{32\varepsilon_0} = \frac{4\pi k^2}{32\varepsilon_0}\frac{R^7}{7} + \frac{4\pi k^2}{32\varepsilon_0}\frac{7R^7}{7} = \frac{\pi k^2R^7}{7\varepsilon_0}
\end{align*}

\newpage

\subsection*{Problem 2.50} 
The electric potential of some configuration is given by the expression

\begin{align*}
    V(r) = A\frac{e^{-\lambda r}}{r}
\end{align*}

Find the electric field $E(r)$, the charge density $\rho(r)$, and the total charge $Q$.

\paragraph{Solution}

The relationship between the electric field and electric potential is given by 

\begin{align*}
    E = -\nabla V
\end{align*}

The potential depends only on the radius in the given formula. Thus, the electric field is given by 

\begin{align*}
    E(r) = -\frac{d}{dr}\left[ A\frac{e^{-\lambda r}}{r} \right] = -A\frac{-\lambda e^{-\lambda r}r - e^{-\lambda r}}{r^2} = Ae^{-\lambda r}\frac{(\lambda r + 1)}{r^2}
\end{align*}

To find the charge density, we recognize that the divergence of the electric field is equal to 

\begin{align*}
    \nabla \cdot E = - \nabla^2V = \frac{\rho}{\varepsilon_0}
\end{align*}

Using the definition of the Laplacian in spherical coordinates we have 

\begin{align*}
    \nabla^2 V &= \frac{1}{r^2}\frac{\partial}{\partial r}\left(r^2 \frac{\partial}{\partial r} \left[ A\frac{e^{-\lambda r}}{r} \right]\right) \\
    \\ 
    &= \frac{1}{r^2}\frac{\partial }{\partial r}\left[  -Ae^{-\lambda r}(\lambda r + 1)\right] \\
    \\
    &= -A \frac{1}{r^2}\left[ -\lambda e^{-\lambda r}(\lambda r + 1) + \lambda e^{-\lambda r}  \right] = A\frac{\lambda^2e^{-\lambda r}}{r}
\end{align*}

\pagebreak

The charge density for $r > 0 $ is then 

\begin{align*}
    \rho = -\varepsilon_0A\frac{\lambda^2e^{-\lambda r}}{r}
\end{align*}

For $r = 0$, the function blows up and we see that there is a singularity. For small $r$ we can expand $Ae^{-\lambda r}/r$ as 

\begin{align*}
    A\frac{e^{-\lambda r}}{r} = A\frac{1}{r} - A\lambda + A\frac{\lambda^2 r}{2} + \cdots
\end{align*}

Since the singularity arises from the $1/r$ term, and by linearity of the $\nabla^2$ operator we have 

\begin{align*}
    \nabla^2 (A\frac{1}{r} + A\lambda ) = A\nabla^2 \frac{1}{r} = -A4\pi \delta^3(\vec{r})  \\
    \implies \rho(\vec{0}) = 4\pi A\varepsilon_0\delta^3(\vec{0})
\end{align*}

Summing these together gives 

\begin{align*}
    \rho({\vec{r}}) = 4\pi A\varepsilon_0\delta^3(\vec{r}) -\varepsilon_0A\frac{\lambda^2e^{-\lambda r}}{r} 
\end{align*}

The total charge is given by integrating through all of space. 

\begin{align*}
    Q = \int_0^{\infty} \left[4\pi A\varepsilon_0\delta^3(\vec{r}) -\varepsilon_0A\frac{\lambda^2e^{-\lambda r}}{r}\right] dr &= 4\pi A\varepsilon_0 \int_0^{\infty} \delta^3(\vec{r})\ dr -\varepsilon_0A\lambda^2 \int_0^{\infty} \frac{e^{-\lambda r}}{r}\ dr \\
    \\
    &= 4\pi A\varepsilon_0 - \varepsilon_0A\lambda^24\pi 
    \frac{1}{\lambda ^2} = 0
\end{align*}

\newpage

\subsection*{Problem 2.51}
Find the potential on the rim of a uniformly charged disk (radius $R$, charge density $\sigma$). 

\paragraph{Solution} The potential at a point $\vec{r}$ due to a surface charge density is given by 

\begin{align*}
    V(\vec{r}) = \frac{1}{4\pi\varepsilon_0}\int \frac{\sigma(\vec{r}')}{\scriptr}\ da
\end{align*}

where $\vec{r}$ is the vector from our reference point to where potential is being measured, $\vec{r}'$ is the vector from point of reference to the charge element $dq$, and $\scriptr = |\vec{r} - \vec{r}'|$. The potential in our situation can be calculated as 

\begin{align*}
    V = \frac{\sigma}{4\pi \varepsilon_0} \int_0^R \int_0^{2\pi} \frac{r}{\sqrt{R^2 + r^2 - 2Rr\cos\theta}} d\theta\ dr
\end{align*}

This integral has no analytical solution, and so instead we can change variables to put the radius of the disc into a $[0,1]$ interval. Let $u = r/R$, then $R du = dr$ and 

\begin{align*}
    V &= \frac{\sigma}{4\pi \varepsilon_0}\int_0^1 \int_0^{2\pi}\frac{R^2u}{R\sqrt{1 + u^2 - 2u\cos\theta}} \ du\ d\theta \\
    \\
    V &= \frac{R\sigma}{4\pi \varepsilon_0}\int_0^1 \int_0^{2\pi}\frac{u}{\sqrt{1 + u^2 - 2u\cos\theta}} \ du\ d\theta
\end{align*}

The double integral can be evaluated numerically and evaluates to approximately 4. The potential is then 
\begin{align*}
    V = \frac{\sigma R}{\pi \varepsilon_0}
\end{align*}

The key to solving this problem was to recognize that it could be transformed in a way that the integral could remain a constant and we could choose $R$ as a parameter.

\newpage

\subsection*{Problem 2.52}
Two infinitely long wires running parallel to the x axis carry uniform charge densities $\lambda$ and $-\lambda$.

\begin{figure}[h]
    \centering
    \includegraphics[width=0.5\linewidth]{infinitely-long-wires.png}
\end{figure}

\paragraph{(a)} Find the potential at any point $(x, y, z)$ using the origin as your reference. 

\paragraph{Solution} Our goal is to determine the potential at any point using the origin as our reference. We can do this by finding the path integral 

\begin{align*}
    V(x, y, z) - V(origin) = -\int_{origin}^{x, y, z} E\cdot dl.
\end{align*}

We can find the electric field by calculating the contribution from each wire. By Gauss's Law, the electric fields point radially outward from each wire. Since they run parallel to the $x-axis$, we can easily identify that $E_x = 0$ for all $(x, y, z)$.  This also implies that $dE/dx = 0$ for all $x, y, z$, so that the strength of the field depends solely on where we are in the $y-z$ plane.\\
\newpage
We draw an example for a point $P = (x, y, z)$ from a viewpoint that looks down the $x$ axis. 

\input{tex_images/parallel-wire-problem}

Using Gauss's law, we know that the magnitude of the electric field a distance $r$ away from an infinitely charged wire is 

\begin{align*}
    E = \frac{\lambda}{2\pi\varepsilon_0}\frac{1}{r}
\end{align*}

The magnitude of the electric fields due to the positively charged wire and negatively charged wire at point $P$ are 

\begin{align*}
    |E^+| &= \frac{\lambda}{2\pi\varepsilon_0}\frac{1}{\sqrt{z^2 + (a - y)^2}} \\
    |E^-| &= \frac{\lambda}{2\pi\varepsilon_0}\frac{1}{\sqrt{z^2 + (a + y)^2}}
\end{align*}

The components of each field can be determined using the angles $\beta$ and $\gamma$. For the field due to the positive wire we have the following. 

\begin{align*}
    \vec{E}^+_y &= -|E^+|\cos{\beta}\bvec{y} = -\frac{\lambda}{2\pi\varepsilon_0}\frac{1}{\sqrt{z^2 + (a - y)^2}} \frac{a - y}{\sqrt{z^2 + (a - y)^2}} \bvec{y}
    = -\frac{\lambda}{2\pi\varepsilon_0}\ \frac{a - y}{z^2 + (a - y)^2}\bvec{y}\\
    \vec{E}^+_z &= |E^+|\sin{\beta}\bvec{z} = \frac{\lambda}{2\pi\varepsilon_0}\frac{1}{\sqrt{z^2 + (a - y)^2}} \frac{z}{\sqrt{z^2 + (a - y)^2}}\bvec{z} 
    = \frac{\lambda}{2\pi\varepsilon_0}\ \frac{z}{z^2 + (a - y)^2}
    \bvec{z}
\end{align*}
\newpage

For the negative wire we have 

\begin{align*}
    \vec{E}^-_y &= -|E^-|\cos{\beta}\bvec{y} = -\frac{\lambda}{2\pi\varepsilon_0}\frac{1}{\sqrt{z^2 + (a + y)^2}} \frac{a + y}{\sqrt{z^2 + (a + y)^2}} \bvec{y}
    = -\frac{\lambda}{2\pi\varepsilon_0}\ \frac{a + y}{z^2 + (a + y)^2}\bvec{y}\\
    \vec{E}^-_z &= -|E^-|\sin{\beta}\bvec{z} = -\frac{\lambda}{2\pi\varepsilon_0}\frac{1}{\sqrt{z^2 + (a + y)^2}} \frac{z}{\sqrt{z^2 + (a + y)^2}} \bvec{z}
    = -\frac{\lambda}{2\pi\varepsilon_0}\ \frac{z}{z^2 + (a + y)^2}\bvec{z}
\end{align*}

We can add the contributions from each wire to obtain the total field. 

\begin{align*}
    \vec{E}_y &= -\frac{\lambda}{2\pi\varepsilon_0}\ \frac{a - y}{z^2 + (a - y)^2}\bvec{y}  -\frac{\lambda}{2\pi\varepsilon_0}\ \frac{a + y}{z^2 + (a + y)^2}\bvec{y} \\ 
    &= - \frac{\lambda}{2\pi \varepsilon_0}\left[ \frac{a - y}{z^2 + (a - y)^2} +  \ \frac{a + y}{z^2 + (a + y)^2} \right] \bvec{y}\\
    \\
    \vec{E}_z &= \frac{\lambda}{2\pi\varepsilon_0}\ \frac{z}{z^2 + (a - y)^2}
    \bvec{z} - \frac{\lambda}{2\pi\varepsilon_0}\ \frac{z}{z^2 + (a + y)^2}\bvec{z}\\
    \\
    &= \frac{\lambda}{2\pi\varepsilon_0} \left[ \frac{z}{z^2 + (a - y)^2} - \frac{z}{z^2 + (a + y)^2} \right]\bvec{z}
\end{align*}

We now perform the integration. Because electric fields are conservative, they are path independent. This means we can get to point $P = (x, y, z)$, where $x$ does not contribution in this case, by moving in the following order $(0, 0, 0) \rightarrow (0, y, 0) \rightarrow (0, y, z)$.  Along the first path, we have $z = 0$, so the integration reduces to only component that we need to integrate is 

\begin{align*}
    \Delta V_y &= -\int_0^y - \frac{\lambda}{2\pi \varepsilon_0}\left[ \frac{a - l}{(a - l)^2} +  \ \frac{a + l}{(a + l)^2} \right] \ dl \\
    \\
    &= \frac{\lambda}{2\pi \varepsilon_0} \int_0^y \left[ \frac{1}{a - l} +  \ \frac{1}{a + l} \right] \ dy \\
    \\
    &= \frac{\lambda}{2\pi \varepsilon_0} \left[ -\ln(a - l) + \ln(a + l) \right]\bigg |^y_0 \\
    \\
    &= \frac{\lambda}{2\pi \varepsilon_0} \ln\left[\frac{a + y}{a - y}\right]
\end{align*}

The integral from $(0, y, 0) \rightarrow (0, y, z)$ is given by 

\begin{align*}
    \Delta V_z &= -\int_0^z \frac{\lambda}{2\pi\varepsilon_0} \left[ \frac{l}{l^2 + (a - y)^2} - \frac{l}{l^2 + (a + y)^2} \right] dl \\
    \\
    &= -\frac{\lambda}{2\pi\varepsilon_0} \int_0^z \left[ \frac{l}{l^2 + (a - y)^2} - \frac{l}{l^2 + (a + y)^2} \right] dl\\
    \\
    &=-\frac{\lambda}{4\pi\varepsilon_0}\left[ \ln(l^2 + (a - y)^2) - \ln(l^2 + (a + y)^2) \right]^z_0\\
    \\
    &=-\frac{\lambda}{4\pi\varepsilon_0}\ln\left[\frac{l^2 + (a - y)^2}{l^2 + (a + y)^2}\right]\bigg |^z_0\\
    \\
    &= -\frac{\lambda}{4\pi\varepsilon_0}\left[ \ln\left[\frac{z^2 + (a - y)^2}{z^2 + (a + y)^2}\right] - \ln\left[\frac{(a - y)^2}{(a + y)^2}\right]\right] \\
    \\
\end{align*}

The total potential is given by the sum of the two integrals. 

\begin{align*}
    \Delta V &= \Delta V_y + \Delta V_z =  \frac{\lambda}{2\pi \varepsilon_0} \ln\left[\frac{a + y}{a - y}\right] -\frac{\lambda}{4\pi\varepsilon_0}\left[ \ln\left[\frac{z^2 + (a - y)^2}{z^2 + (a + y)^2}\right] - \ln\left[\frac{(a - y)^2}{(a + y)^2}\right]\right] \\
    \\
    &= \frac{2\lambda}{4\pi \varepsilon_0} \ln\left[\frac{a + y}{a - y}\right] -\frac{\lambda}{4\pi\varepsilon_0}\ln\left[\frac{z^2 + (a - y)^2}{z^2 + (a + y)^2}\right] + \frac{\lambda}{4\pi\varepsilon_0}\ln\left[\frac{(a - y)^2}{(a + y)^2}\right]\\
    \\
    &= \frac{\lambda}{4\pi \varepsilon_0} \ln\left[\frac{(a + y)^2}{(a - y)^2}\right] -\frac{\lambda}{4\pi\varepsilon_0}\ln\left[\frac{z^2 + (a - y)^2}{z^2 + (a + y)^2}\right] + \frac{\lambda}{4\pi\varepsilon_0}\ln\left[\frac{(a - y)^2}{(a + y)^2}\right]\\
    \\
    &= -\frac{\lambda}{4\pi\varepsilon_0}\ln\left[\frac{z^2 + (a - y)^2}{z^2 + (a + y)^2}\right]
\end{align*}

\end{document}